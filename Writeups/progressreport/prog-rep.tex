\documentclass[10pt, onecolumn, draftclsnofoot, letterpaper, compsoc]{IEEEtran}

\usepackage{graphicx}
\usepackage{amssymb}
\usepackage{amsmath}
\usepackage{amsthm}
\usepackage{alltt}
%\usepackage{float}
\usepackage{color}
\usepackage{url}
\usepackage{minted}

\renewcommand*\contentsname{Table of Contents} % Rename ToC

% Temp title and author
\title{Progress Report}
\author{Totality AweSun \\
		Bret~Lorimore, Jacob~Fenger, George~Harder \\
		\textit{\today \\
		CS 461 - Fall 2016}}

\begin{document}

\maketitle

\begin{abstract}

Text

\end{abstract}

\newpage

\tableofcontents

\newpage

%%%%%%%%%%%%%%%%%%%%%%%%
%   Project Overview   %
%%%%%%%%%%%%%%%%%%%%%%%%
\section{Project Overview}

The North American Solar Eclipse 2017 Senior Capstone project is partnership
with Google to build a set of applications that will assist the development of
the Eclipse Megamovie Project. The overall project has been broken down into
three components: the eclipse image processor, the image processor manager, and
the solar eclipse simulator. Each will be individually outlined in the sections
below.

\subsection{Image Processor}

The image processor’s primary activity is to quickly and consistently identify
images of an eclipse at totality. The Eclipse Megamovie project will be
collecting thousands of images from photographers around the country, and the
image processor needs to identify the images of the eclipse at totality so that
these can then be stitched into a timelapse movie. In order to make the
stitching as easy as possible for the Eclipse Megamovie team, the image
processor will add metadata to each processed image that includes spatial
information about where the image was taken along the path of totality and
temporal information about how far into totality the eclipse is.


The purpose of the Image Processor is not to process many images as quickly as
possible. Instead, our goal is to be able to consistently and accurately process
a single image at a time. As such, the image processor will fit into the larger
project as an executable file that is called by the Image Processor Manager.
This allows us to focus the image processor solely on a single goal, and leave
parallelization and deployment to a different piece of the project.

\subsection{Image Processor Manager}

The image processor manager will be a Python application responsible for managing the image processor
application. This includes collecting images from Google Cloud for the image processor to process,
invoking the image processor with these images as input, and collecting the output of the image
processor and uploading it to Google Cloud. The image processor and image processor manager will be
deployed together in a single docker container to Google Container Engine Clusters (of VMs). \\

An important role of the image processor manager is that it will be responsible for ensuring that
the compute resources on the host VMs are as saturated as possible. This means invoking multiple
image processor processes concurrently, while at the same time downloading the next images to be
processed and uploading the already processed images. The image processor manager will achieve
this parallelism through process based concurrency in Python, as in Python, thread based concurrency
is throttled by the global interpreter lock (GIL). We chose to use Python for this application as it
will be much simpler to write in Python and we can sidestep any concurrency issues by using process
based parallelism as mentioned above. \\

\subsection{Eclipse Simulator}

The eclipse simulator will be an independent JavaScript module that can easily
be added to the existing Eclipse Megamovie webpage. This simulator will allow
users to “preview” the eclipse. It will be a 2D depiction of what the solar
eclipse in 2017 could look like given a certain location. Users will be able
to interact with a time slider that will simulate the eclipse in a time
window spanning from 12 hours before the eclipse to 12 hours after it.

To help with the eclipse ephemeris computations, we will be using an external
JavaScript library called Ephemeris. For the front end view for the simulator,
we will be utilizing HTML5 SVG. We plan to implement a model-view-controller
architecture for controlling the states of each component as well as handling
the interactions. This architecture was chosen due to the ability to easily
exchange a component without altering the whole design of the system. For
example, if one wanted to create a whole new front end for the simulator,
they would not need to rewrite the model or controller component of the system.
They would simply need to ensure that the new view component can handle the
interactions with the controller module.


%%%%%%%%%%%%%%%%%%%%%%%%
%   Current Status     %
%%%%%%%%%%%%%%%%%%%%%%%%
\section{Current Status of the Project}

As a whole this project has moved from the pre-planning stages into the earliest
development phase. The image processor and the image processor manager are both
fully designed, while the eclipse simulator is farther along and has some existing proof of
concept code.

\subsection{Image Processor}

This part of the project is still in the design and planning phase, development
work has not begun. However, we have a clear design plan and have identified
several tools and technologies that we will use to build the image processor.

The image processor will use OpenCV and its built in Hough Transform as well as
an existing algorithm that we plan to modify and improve upon in order to
identify eclipses in images. We have chosen to use C++ to build the image
processor because it gives us better control over the speed at which this
application runs while still providing us with the ability to leverage the
OpenCV API. We have also clearly defined the inputs and outputs of the in order
to allow the parts of the system that interact with the image processor to
proceed independently with their development.

\subsection{Image Processor Manager}

Development has yet to start on the image processor manager, but we have a very good idea of how we
are going to build it. Additionally, we will have access to code that Bret wrote during his internship
this summer that we will be able to repurpose to handle large parts of the interactions with Google Cloud.

\subsection{Eclipse Simulator}

As it currently stands, progress has been made to create an initial working
demo of the simulator. We have worked with the Ephemeris library to see how
accurate its computations are. There was some variance when comparing the
values it returned to other 2017 solar eclipse predictions, but we have
decided that the variance is not large enough to warrant worry. Work has
been done to create a front end view for the simulator. Eventually Google
will be providing the sprites that the simulator will use, but we are currently
just using simple SVG circles to simulate the Sun and Moon. We also have the
code to render these Sun and Moon SVGs on the eclipse simulator window given
their altitude and azimuth (vertical/horizontal angular coordinates). We will
continue working on the eclipse simulator over winter break.

%%%%%%%%%%%%%%%%%%%%%%%%
%   Problems           %
%%%%%%%%%%%%%%%%%%%%%%%%
\section{Problems and Possible Solutions}

Around week 8 of the term, we found that some results differed in the Ephemeris
library calculations and eclipse predictions we found elsewhere. We brought
this issue up with our client, and he said to continue progress on the
simulator as the variance we were seeing was not very significant for the
simulator. Once further progress has been made on the simulator, it may be
necessary to dig into the Ephemeris code base to determine any problems.

%%%%%%%%%%%%%%%%%%%%%%%%
%   Code               %
%%%%%%%%%%%%%%%%%%%%%%%%
\section{Interesting Code}

Below are some interesting sections of the Eclipse Simulator view code that we have so far. 
The code shown below focuses on rendering the Sun and Moon on the screen given their
altitude and azimuth. \\

\begin{minted}{javascript}
// EclipseSimulator namespace
var EclipseSimulator = {

   // [...]

    View: function()
    {
        this.window   = $('#container').get(0);
        this.controls = $('#controls').get(0);
        this.sun      = $('#sun').get(0);
        this.moon     = $('#moon').get(0);

      // temp radius values
        this.sunpos  = {x: 50, y: 50, r: 2 * Math.PI / 180};
        this.moonpos = {x: 25, y: 25, r: 2 * Math.PI / 180};

        // Field of view in radians
        this.fov = {x: 80 * Math.PI / 180, y: 80 * Math.PI / 180};

        // Center of frame in radians
        this.az_center = 0 * Math.PI / 180;
    },

    // [...]

    // Convert degrees to radians
    deg2rad: function(v)
    {
        return v * Math.PI / 180;
    },

    // Convert radians to degrees
    rad2deg: function(v)
    {
        return v * 180 / Math.PI;
    },

    // Convert a to be on domain [0, 2pi)
    normalize_rad: function(a)
    {
        var pi2 = Math.PI * 2;
        return a - (pi2 * Math.floor(a / pi2));
    },

    // Compute positive distance in radians between two angles
    rad_diff: function(a1, a2)
    {
        a1 = EclipseSimulator.normalize_rad(a1);
        a2 = EclipseSimulator.normalize_rad(a2);

        var diff = a1 > a2 ? (a1 - a2) : (a2 - a1);

        return diff > Math.PI ? (2 * Math.PI) - diff : diff;
    },

    // Determine if angle a is greater than angle b
    // That is, if b < a <= (b + pi)
    rad_gt: function(a, b)
    {
        a = EclipseSimulator.normalize_rad(a);
        b = EclipseSimulator.normalize_rad(b);

        a = EclipseSimulator.normalize_rad(a - b);
        b = 0;

        return a > b && a <= Math.PI;
    },

    // [...]
};

// [...]


EclipseSimulator.View.prototype.get_x_percent_from_az = function(az, radius)
{
    var dist_from_center = Math.sin(az - this.az_center);
    var half_fov_width   = Math.sin(this.fov.x / 2);

    if (this.az_out_of_view(az, radius))
    {
        // Just move the body way way off screen
        dist_from_center += this.fov.x * 10;
    }

    return 50 + (50 * dist_from_center / half_fov_width);
}

// This may need some re-visiting... the current computations imply a field of
// view of 2*fov.y
// Compute y coordinate in simulator window as a percentage of the window
// height
// Assumes altitude is <= (pi/2)
EclipseSimulator.View.prototype.get_y_percent_from_alt = function(alt)
{
    var height     = Math.sin(alt);
    var fov_height = Math.sin(this.fov.y);

    return 100 * height / fov_height;
}

EclipseSimulator.View.prototype.az_out_of_view = function(az, radius)
{
    var bound = this.az_center + (this.fov.x / 2);
    var dist  = EclipseSimulator.rad_diff(bound, az);
    // Body off screen to the right
    if (EclipseSimulator.rad_gt(az, bound) && dist > radius)
    {
        return true;
    }

    bound = this.az_center - (this.fov.x / 2);

    dist  = EclipseSimulator.rad_diff(bound, az);

    // Body off screen to the left
    if (EclipseSimulator.rad_gt(bound, az) && dist > radius)
    {
        return true;
    }

    return false;
}

// [...]
\end{minted}

%%%%%%%%%%%%%%%%%%%%%%%%
%   Weekly Summary     %
%%%%%%%%%%%%%%%%%%%%%%%%
\section{Week by Week Summary of Group Activities}

\subsection{Week 1}

    \begin{itemize}

    \item Go to class

    \end{itemize}

\subsection{Week 2}

    \begin{itemize}

    \item Worked on initial draft of problem statement.

    \item Established a LaTeX workflow.

    \end{itemize}

\subsection{Week 3}

    \begin{itemize}

    \item Got feedback from sponsor and Kirstin on initial problem statement 
    draft. Our sponsor was happy with our initial draft. Kirstin informed us that 
    our problem statement was actually a bit too verbose and covered a lot of 
    material that was more relevant to the technology review.

    \end{itemize}

\subsection{Week 4}

    \begin{itemize}

    \item Had an initial meeting with Vee, our TA.

    \item Researched rotation matrices and quaternions at the suggestion of our 
    sponsor. He pointed us to this material as being relevant to our rotating 
    images, but likely more involved than actually necessary. He still encouraged 
    us to research the material as he thought it was interesting and good to 
    know. It was!

    \end{itemize}

\subsection{Week 5}

    \begin{itemize}

    \item Created initial requirements document draft.

    \end{itemize}

\subsection{Week 6}

    \begin{itemize}

    \item Reviewed requirements document with project sponsor.

    \item Followed up on new information we received from our client in week 5, 
    specifically that we would have access to GPS/timestamp EXIF info in the 
    image processor. We considered this and proposed a scheme to incorporate it 
    into the image processor that our client was very happy with.

    \end{itemize}

\subsection{Week 7}

    \begin{itemize}

    \item Our sponsor added an additional component to the project, the image 
    processor manager. Previously this had been outside the scope of the project. 
    We were initially hesitant about taking this on, but realized that there was 
    some relevant code from Bret’s internship over the summer that we can 
    repurpose for this component of the project.

    \end{itemize}

\subsection{Week 8}

    \begin{itemize}

    \item Began work on the eclipse simulator. We ran into some issues computing 
    accurate sun/moon positions using JavaScript. The first library we were 
    using, SunCalc was not producing accurate results. After switching to the 
    Ephemeris library, we started seeing better results.

    \end{itemize}

\subsection{Week 9}

    \begin{itemize}

    \item Met as a group and ironed out a high level design for the various 
    components of the project. 

    \item Individually began working on our respective design document pieces.

    \item Continued to see accuracy issues with eclipse calculations in 
    JavaScript, even after switching to Ephemeris. We spoke to our sponsor about 
    this and he recommended moving on for the time being, as he did not think 
    these computations were off enough to warrant concern, or even render them 
    unusable.

    \end{itemize}

\subsection{Week 10}

    \begin{itemize}

    \item Completed and turned in design document, we struggled a bit 
    interpreting the IEEE 1016 standard for the design document.

    \item Began work on the view code for the eclipse simulator.

    \end{itemize}

\section{Retrospectives}

\begin{table}[h]
    \centering
    \begin{tabular}{|p{.3\linewidth}|p{.3\linewidth}|p{.3\linewidth}|}

    \cline{3-3}

    \hline \textbf{Positives} & \textbf{Deltas} & \textbf{Actions} \\ \hline

    Image Processor design completed &  & \\ \hline
    Image Processor Manager design completed &  & \\ \hline
    Began work on Eclipse Simulator & User interaction & Implement buttons to alter sun/moon positions \\ \hline

    Have access to Solar Eclipse Image Standardisation and Sequencing algorithm, research paper and code
    & Existing code is written in IDL and the algorithm does not completely meet our needs at this point in time
    & Port code to C++ and modify algorithm to meet our needs \\ \hline

    Have reusable code from Bret’s internship that is relevant to both the Image Processor and Image Processor Manager
    & Need to work on re-purposing this code to fit our needs and incorporate into our project
    & Need to work with sponsor to get this code open sourced \\ \hline

    Sponsor impressed by proposed usage of EXIF GPS data in Image Processor &  & \\ \hline


    \end{tabular}
\end{table}

\end{document}
