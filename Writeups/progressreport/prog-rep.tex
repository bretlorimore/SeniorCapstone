\documentclass[10pt, onecolumn, draftclsnofoot, letterpaper, compsoc]{IEEEtran}

%\usepackage{cite}
\usepackage{float}
\usepackage{hyperref}
%\usepackage{enumitem}
\usepackage{graphicx}

\renewcommand*\contentsname{Table of Contents} % Rename ToC

% Temp title and author
\title{Progress Report}
\author{Totality AweSun \\
		Bret~Lorimore, Jacob~Fenger, George~Harder \\
		\textit{\today \\
		CS 461 - Fall 2016}}

\begin{document}

\maketitle

\begin{abstract}

Text

\end{abstract}

\newpage

\tableofcontents

\newpage

%%%%%%%%%%%%%%%%%%%%%%%%
%   Project Overview   %
%%%%%%%%%%%%%%%%%%%%%%%%
\section{Project Overview}

Text

\subsection{Image Processor}

Text

\subsection{Image Processor Manager}

The image processor manager will be a Python application responsible for managing the image processor
application. This includes collecting images from Google Cloud for the image processor to process, 
invoking the image processor with these images as input, and collecting the output of the image 
processor and uploading it to Google Cloud. The image processor and image processor manager will be 
deployed together in a single docker container to Google Container Engine Clusters (of VMs). \\

An important role of the image processor manager is that it will be responsible for ensuring that 
the compute resources on the host VMs are as saturated as possible. This means invoking multiple 
image processor processes concurrently, while at the same time downloading the next images to be 
processed and uploading the already processed images. The image processor manager will achieve 
this parallelism through process based concurrency in Python, as in Python, thread based concurrency 
is throttled by the global interpreter lock (GIL). We chose to use Python for this application as it 
will be much simpler to write in Python and we can sidestep any concurrency issues by using process 
based parallelism as mentioned above. \\

\subsection{Eclipse Simulator}

Text


%%%%%%%%%%%%%%%%%%%%%%%%
%   Current Status     %
%%%%%%%%%%%%%%%%%%%%%%%%
\section{Current Status of the Project}

Text

\subsection{Image Processor}

Text

\subsection{Image Processor Manager}

Development has yet to start on the image processor manager, but we have a very good idea of how we 
are going to build it. Additionally, we will have access to code that Bret wrote during his internship 
this summer that we will be able to repurpose to handle large parts of the interactions with Google Cloud.

\subsection{Eclipse Simulator}

Text

%%%%%%%%%%%%%%%%%%%%%%%%
%   Problems           %
%%%%%%%%%%%%%%%%%%%%%%%%
\section{Problems and Possible Solutions}

Text

%%%%%%%%%%%%%%%%%%%%%%%%
%   Code               %
%%%%%%%%%%%%%%%%%%%%%%%%
\section{Interesting Code}

Code

%%%%%%%%%%%%%%%%%%%%%%%%
%   Weekly Summary     %
%%%%%%%%%%%%%%%%%%%%%%%%
\section{Week by Week Summary of Group Activities}

\subsection{Week 1}

\begin{itemize}

\item Text

\end{itemize}

\subsection{Week 2}

\begin{itemize}

\item Text

\end{itemize}

\subsection{Week 3}

\begin{itemize}

\item Text

\end{itemize}

\subsection{Week 4}

\begin{itemize}

\item Text

\end{itemize}

\subsection{Week 5}

\begin{itemize}

\item Text

\end{itemize}

\subsection{Week 6}

\begin{itemize}

\item Text

\end{itemize}

\subsection{Week 7}

\begin{itemize}

\item Text

\end{itemize}

\subsection{Week 8}

\begin{itemize}

\item Text

\end{itemize}

\subsection{Week 9}

\begin{itemize}

\item Text

\end{itemize}

\subsection{Week 10}

\begin{itemize}

\item Text

\end{itemize}

\section{Retrospectives}

\begin{table}[h]
    \centering
    \begin{tabular}{|p{.3\linewidth}|p{.3\linewidth}|p{.3\linewidth}|}

    \cline{3-3}

    \hline \textbf{Positives} & \textbf{Deltas} & \textbf{Actions} \\ \hline

    Lorem & Lorem & Lorem \\ \hline

    \end{tabular}
\end{table}

\end{document}