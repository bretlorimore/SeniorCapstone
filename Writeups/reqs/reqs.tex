\documentclass[10pt, onecolumn, draftclsnofoot, letterpaper]{IEEEtran}

\usepackage{cite}
\usepackage{hyperref}


%%%%%%%%%%%%%%%%%%%%%%%%%%%%%%%%%%%%%%%%%%%%%%%%%%%%%
% Macro for the signatures at the end                %
\newcommand*{\SignatureAndDate}[1]{%
    \par\noindent\makebox[2.5in]{\hrulefill} \hfill\makebox[2.0in]{\hrulefill}%
    \par\noindent\makebox[2.5in][l]{#1}      \hfill\makebox[2.0in][l]{Date}%
}%

% Temp title and author
\title{Requirements}
\date{\today} % Somehow this isn't working..
\author{Totality AweSun \\
		Bret~Lorimore, Jacob~Fenger, George~Harder \\
		\textit{October 28th, 2016 \\
		CS 461 - Fall 2016}} 

\begin{document}
\maketitle

\section{Eclipse Simulator Requirements}

\begin{enumerate}
	\item The simulator is a standalone JavaScript module that can 
	easily be included on an existing webpage.

	\item Users can select the location from which to simulate the 
	eclipse. This can be entered at any point while using the 
	simulator.
	\begin{enumerate}
		\item Location can be entered as: lat/long, address, zip
		code, city name, state name.
		\item Initial simulator location can be programmatically 
		set as initialization parameter
	\end{enumerate}

	\item Users will be able to adjust the simulator time from 12 
	before the eclipse to 12 hours after it.
	\begin{enumerate}
		\item Time can be advanced via a slider, or clickable 
		buttons.
	\end{enumerate}

	\item All simulator resources will load in less than= 500ms given a 
	1-10 Mbps internet connection. Once this is achieved file sizes
	will be optimized to achieve the fastest load time possible.

	\item Simulator will display the local time that is being 
	displayed, e.g. there is a well defined time associated with 
	the user selecting Corvallis, Oregon as their location and a 
	simulator time of -3:13 (3 hours 13 minutes) before the 
	eclipse. This time should be displayed on the simulator.

	\item Solar/lunar placement will be scientifically accurate for
	the given location and time. This accuracy will apply to edge 
	cases as well, where the location is say, not in the path of 
	totality, or on the opposite side of the world as the eclipse.
	In the latter situation, the simulator would have to shift to
	a night time display during the eclipse.
\end{enumerate}

\section{Image Pre-Processor Requirements}

\begin{enumerate}
	\item The image pre-processor will be compatible with Ubuntu 16.04 and will include a script that to install all dependencies/
	build the binary.

	\item The application will accept the following input as 
	command line arguments:
	\begin{enumerate}
		\item Required: image\_list\_file
		\begin{enumerate}
			\item Absolute or relative (to the directory the binary 
			was invoked from) path to file containing a list of image 
			file names with no directory.
		\end{enumerate}

		\item Required: output\_dir
		\begin{enumerate}
			\item Directory to write output files to.
		\end{enumerate}

		\item Optional: image\_path\_prefix
		\begin{enumerate}
			\item Absolute or relative (to the directory the 
			binary was invoked from) path to prepend to each image 
			filename in image\_file\_list. Defaults to “./”.
		\end{enumerate}
	\end{enumerate}

	\item The application will accept JPEG (.jpeg/.jpg) and PNG (.
	png) image files.
	\begin{enumerate}
		\item Images of an invalid format will be disregarded and an error message will be written to stderr.
	\end{enumerate}

	\item Invalid JPEG and PNG files (e.g. cannot be opened by 
	OpenCV, width/height equal to 0px, etc.) will be ignored and 
	an error message will be written to stderr.

	\item The application will classify the input images as being 
	one of the following types:
	\begin{enumerate}
		\item FULL\_DISK
		\begin{enumerate}
			\item Image of an unobscured solar disk.
		\end{enumerate}

		\item TOTALITY
		\begin{enumerate}
			\item Image of a total solar eclipse.
		\end{enumerate}

		\item CRESCENT
		\begin{enumerate}
			\item Image of a partially eclipsed sun, creating a 
			“crescent” shape.
		\end{enumerate}

		\item DIAMOND\_RING
		\begin{enumerate}
			\item Image of a nearly fully eclipsed sun where there
			is one “hot spot” on the sun's perimeter. This hot 
			spot along with the sun's perimeter have the shape of
			a diamond ring.
		\end{enumerate}
	\end{enumerate}

	\item The preprocessor will sort the images into five separate 
	bins based on the relative delta of the position of the center 
	of the sun and the moon. This delta will be a signed value 
	based on the sun's position, i.e. if the moon is to the left 
	of the sun (in the cropped/rotated image) the relative delta 
	will be negative and conversely, if the moon is to the right 
	of the sun the delta value will be positive.

	\item The application will crop the images to a standard size 
	with the sun centered in them.

	\item The application will rotate DIAMOND\_RING and CRESCENT 
	type images so that they are aligned horizontally, as 
	described by Krista et al.

	\item The application will write the following output to the output\_dir directory:
	\begin{enumerate}
		\item image\_transformation.txt
		\begin{enumerate}
			\item File containing one line per image processed 
			with the following values (comma separated):
			\begin{enumerate}
				\item processed\_image
				\item image\_type
				\item rot\_angle
				\item crop\_topl\_x
				\item crop\_topl\_y
				\item crop\_botr\_x
				\item crop\_botr\_y
				\item rel\_center\_offset - optional, only included for crescent type images
			\end{enumerate}
		\end{enumerate}

		\item Pre-processed image files
			\begin{enumerate}
				\item images/*\_pp.[png|jpeg|jpg]
			\end{enumerate}
	\end{enumerate}

	\item The application will format all log messages as follows:
	\begin{enumerate}
		\item img\_preproc:level:timestamp:”message”
		\begin{enumerate}
			\item Values of level: ERROR/WARNING/INFO/DEBUG
			\item Value of timestamp: current timestamp
			\item Value of message: specific logging message
		\end{enumerate}
	\end{enumerate}
\end{enumerate}


\vspace{20mm}
\noindent \SignatureAndDate{David Konerding, Project Sponsor}
\vspace{8mm}
\noindent \SignatureAndDate{Bret Lorimore}
\vspace{8mm}
\noindent \SignatureAndDate{George Harder}
\vspace{8mm}
\noindent \SignatureAndDate{Jacob Fenger}

\end{document}
