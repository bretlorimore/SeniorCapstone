\documentclass[10pt, onecolumn, draftclsnofoot, letterpaper]{IEEEtran}

\usepackage{cite}
\usepackage{hyperref}


%%%%%%%%%%%%%%%%%%%%%%%%%%%%%%%%%%%%%%%%%%%%%%%%%%%%%
% Macro for the signatures at the end                %
\newcommand*{\SignatureAndDate}[1]{%
    \par\noindent\makebox[2.5in]{\hrulefill} \hfill\makebox[2.0in]{\hrulefill}%
    \par\noindent\makebox[2.5in][l]{#1}      \hfill\makebox[2.0in][l]{Date}%
}%

% Temp title and author
\title{2017 Solar Eclipse}
\date{\today} % Somehow this isn't working..
\author{Bret~Lorimore, Jacob~Fenger, George~Harder \\
		\textit{October 26th, 2016 \\
		CS 461 - Fall 2016}} 

\begin{document}
\maketitle

\begin{abstract}

On August 21, 2017 a total solar eclipse will pass over the United States. The
path of totality will stretch from Oregon to South Carolina. There has not been
a total solar eclipse like this, crossing the country from coast to coast, since
the eclipse of 1918. The Eclipse Megamovie Project is a collaboration between
Google and scientists from UC Berkeley and several other institutions with the
aim of compiling a large dataset of eclipse observations. Acquiring coronal data
is of particular interest as the corona is not normally visible from Earth.
Specifically, the project will crowdsource photos of the eclipse from
photographers at various locations along the path of totality. These images will
be aligned spatially and temporally and stitched into a unique movie that shows
the eclipse over a period of 1.5 hours as it passes across the United States.
Additionally, the complete photo dataset will be open sourced so that
independent researchers may do their own analysis.

Google will contribute applications providing, among other things, backend image
processing, photo upload capabilities, and static informational content. This
senior capstone project will consist of two distinct sub-projects, specifically,
improving/implementing an image processing algorithm facilitating the
classification and spatial/temporal alignment of solar eclipse images before
they are stitched into a movie, and a location-based eclipse simulator.

\end{abstract}

\newpage % End the current page

\section{Problem Definition}

\subsection{Eclipse Image Pre-Processor}

Solar eclipses offer researchers a rare chance to observe the Sun's corona.
Normally only visible with highly specialized instrumentation, it becomes easily
visible during a solar eclipse. The total solar eclipse in 2017 will offer solar
researchers a unique opportunity to enlist the help of photographers across the
United States to gather images of the Sun and its corona. In preparation,
scientists are collaborating with Google to build the infrastructure to collect
these eclipse observations and make them available to researchers.

In addition to collecting these raw images, one of the focuses of the Eclipse
Megamovie project is to stitch them into a movie of the eclipse as it passes
over the US. This data will enable researchers to observe the Sun's corona over
an unprecedented time window. In order to stitch these images into a movie, they
must be ordered temporally and resized/aligned spatially. The combination of a
wide variety of locations and angles at which the crowdsourced images will be
taken, the lack of reliable GPS and time-stamp data on most modern DSLR cameras,
and the sheer volume of images that will be collected makes this a non-trivial
problem.

\subsection{Eclipse Simulator}

The eclipse's path of totality is small relative to the continental U.S. As
such, it is of interest to photographers and the public, what the eclipse will
look like from a given location. One of the primary goals of the Eclipse
Megamovie website \href{http://eclipsemega.movie}{eclipsemega.movie}, is to give
users access to information about this specific eclipse and eclipses in general.
Enabling users to “preview” the eclipse will increase engagement before the
eclipse and enable users to better select a viewing location.

\newpage

\section{Proposed Solution}

\subsection{Eclipse Image Pre-Processor}

The eclipse image classifier application will ingest photos of the eclipse and
align them spatially and temporally so that they are ready to be stitched into
movies. With images sourced from photographers all with different camera
equipment/configurations and with no guarantees of accurate timestamp and GPS
coordinate data, this problem is challenging.

We plan to create a standalone C++ application using existing image processing
implementations, including OpenCV and other libraries wherever possible.
Specifically, this application will ingest eclipse images, perform the
processing described above, and export them along with all meta-data required to
compile them into movies. This application will not perform any of the stitching
itself, nor will it interface with Google Cloud Storage, where the eclipse
images are stored - it will simply be a binary that can be invoked by some
larger system developed by Google.

\subsection{Eclipse Simulator}

To address the desire for users to be able to “preview” the eclipse, we will
implement a JavaScript eclipse simulator as a standalone widget. It will be
designed in a stylized, 2D manner. The visual design will be provided by Google
along with all the visual assets necessary to build it - including images, SVG
graphics, etc. The simulator will incorporate a time slider that allows users to
simulate the eclipse in a time window spanning from 12 hours before the eclipse
to 12 hours after it. As users drag the time slider, the eclipse will animate in
a scientifically accurate fashion.

\newpage

\section{Performance Metrics}

\subsection{Eclipse Image Pre-Processor}

Given the large quantity of images users will upload to the site, each image
must be processed accurately and in a reasonable amount of time. Each image
should not take more than 5 seconds for pre-processing. While one second per
image is ideal, quality is of high value for  this project and we will likely
face a performance-quality tradeoff.

From a quality perspective, the pre-processor must identify solar/lunar disks
with 95-98\% accuracy. It must also be able to classify images as being of
full-disk/crescent/diamond-ring/total eclipses with 95\% accuracy. These
accuracy metrics will be computed from a set of hand labeled, golden data.
Achieving these accuracy metrics is key to ensure the movie that is exported
is smooth and accurate.

\subsection{Eclipse Simulator}

There are several metrics that will be used to evaluate whether our eclipse
simulator a useable solution. These metrics deal with timing and the speed at
which the simulator responds to user interactions. First, we want the simulator
to load within 150 milliseconds. This quick load time makes users more likely to
use our simulator and gives them a positive impression of the site. Next, our
goal is for the simulator to run at 60 frames per second (fps). This metric is
slightly hardware dependant. As such, frame rates down to 30fps are acceptable
on slow hardware. Frame rates below 30fps are unacceptable.

\vspace{20mm}

\noindent \SignatureAndDate{David Konerding, Project Sponsor}
\vspace{8mm}
\noindent \SignatureAndDate{Bret Lorimore}
\vspace{8mm}
\noindent \SignatureAndDate{George Harder}
\vspace{8mm}
\noindent \SignatureAndDate{Jacob Fenger}

\end{document}

