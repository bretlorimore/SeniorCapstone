\documentclass[10pt, onecolumn, draftclsnofoot, letterpaper, compsoc]{IEEEtran}

\usepackage{cite}
\usepackage{hyperref}
\usepackage[normalem]{ulem}
\usepackage{enumitem}
\usepackage{graphicx}

\graphicspath{ {images/} }

% Colored Text
\usepackage[usenames, dvipsnames]{color}

%%%%%%%%%%%%%%%%%%%%%%%%%%%%%%%%%%%%%%%%%%%%%%%%%%%%%
% Macro for the signatures at the end                %
\newcommand*{\SignatureAndDate}[1]{%
    \par\noindent\makebox[2.5in]{\hrulefill} \hfill\makebox[2.0in]{\hrulefill}%
    \par\noindent\makebox[2.5in][l]{#1}      \hfill\makebox[2.0in][l]{Date}%
}%

\renewcommand*\contentsname{Table of Contents} % Rename ToC

\newcommand{\myindent}{\hspace{\oldparindent}}

% Temp title and author
\title{Requirements}
\date{\today} % Somehow this isn't working..
\author{Totality AweSun \\
		Bret~Lorimore, Jacob~Fenger, George~Harder \\
		\textit{\today \\
		Senior Capstone, Oregon State University}}

\begin{document}

\setlist[itemize]{topsep=1pt} % EDIT LISTS

\maketitle

\begin{abstract}

On August 21, 2017 a total solar eclipse will pass over the United States. The
path of totality will stretch from Oregon to South Carolina. There has not been
a total solar eclipse like this, crossing the country from coast to coast, since
the eclipse of 1918. The Eclipse Megamovie Project is a collaboration between
Google and scientists from UC Berkeley and several other institutions with the
aim of compiling a large dataset of eclipse observations. Acquiring coronal data
is of particular interest as the corona is not normally visible from Earth.
Specifically, the project will crowdsource photos of the eclipse from
photographers at various locations along the path of totality. These images will
be aligned spatially and temporally and stitched into a unique movie that shows
the eclipse over a period of 1.5 hours as it passes across the United States.
Additionally, the complete photo dataset will be open sourced so that
independent researchers may do their own analysis.

Google will contribute applications providing, among other things, backend image
processing, photo upload capabilities, and static informational content. This
senior capstone project will consist of three distinct sub-projects, specifically,
implementing/modifying an image processing algorithm facilitating the
classification and alignment of solar eclipse images before they are stitched into
a movie, a developer pipeline, and a location-based eclipse simulator.

\end{abstract}

\vspace{10mm}
\noindent \SignatureAndDate{David Konerding, Project Sponsor}
\vspace{8mm}
\noindent \SignatureAndDate{Bret Lorimore}
\vspace{8mm}
\noindent \SignatureAndDate{George Harder}
\vspace{8mm}
\noindent \SignatureAndDate{Jacob Fenger}

\newpage

\tableofcontents

\newpage
%%%%%%%%%%%%%%%%%%%%%%%%%%%%%%%%%
%  SECTION 1: INTRODUCTION
%%%%%%%%%%%%%%%%%%%%%%%%%%%%%%%%%%
\section{Introduction}

\subsection{Purpose}
The purpose of this software requirements specification (SRS) is to describe in
detail the Eclipse Image Processor, Eclipse Image Processor Developer Pipeline,
and the Eclipse Simulator that our group will
produce. By writing these requirements down and agreeing to them with our
sponsor both parties will have a clear understanding of what the finished
product will be and what it will be able to do. The intended audience for this
SRS is our sponsor, the Senior Capstone Instruction Team, and ourselves.

\subsection{Scope}
We are producing three products: an Eclipse Image Processor, an Eclipse
Image Processor Developer Pipeline, and an Eclipse
Simulator. The Eclipse Image Processor will ingest images and identify the sun
and moon.
The Eclipse Image Developer Pipeline
will download images to be processed by the image processor, build and run the
image processor on these images, generate an HTML document with the results,
and upload the processed images and HTML document to Google Cloud Storage. The Eclipse
Simulator will provide users with a 2D visual representation of the eclipse from a
specified location within a 3 hour time range.

\subsection{Definitions, acronyms, and abbreviations}

	\textbf{Eclipse Megamovie Project:}
	The Eclipse Megamovie Project is a collaboration between Google
	and scientists from Berkeley and several other institutions with the
	aim of collecting large quantities of observations of the solar eclipse
	that will pass over the United States on August 21, 2017. The project
	will crowdsource photos of the eclipse from photographers at various
	points along the path of totality. \\

	\noindent \textbf{JPEG/JPG:}
	JPEG is a lossy compression technique for images. When we refer
	to JPEG/JPG files in this document we are referring to image files
	compressed in this method with the .jpeg or .jpg file extension. \\

	\noindent \textbf{PNG:}
	PNG refers to the Portable Network Graphics image file format.
	Images in the PNG format are frequently referred to as "PNGs" and are
	saved with the .png file extension.


\subsection{References}

This SRS makes reference to a report by Larisza D. Krista and Scott W.
McIntosh titled "The Standardisation and Sequencing of Solar Eclipse Images for
the Eclipse Megamovie Project." This technical report was produced as a
collaboration between scientists at the University of Colorado at Boulder, the
National Center for Atmospheric Research and the National Oceanic and
Atmospheric Administration. This paper can be found on arxiv.org.

\subsection{Overview}

The remainder of this SRS contains an overall description of the Eclipse Image
Processor, Eclipse Image Processor Developer Pipeline, and Eclipse Simulator
systems in section 2. Following these descriptions are specific requirements
for the systems in section 3.

%%%%%%%%%%%%%%%%%%%%%%%%%%%%%%%%%%%
%  SECTION 2: OVERALL DESCRIPTION
%%%%%%%%%%%%%%%%%%%%%%%%%%%%%%%%%%%%%%%
\newpage
\section{Overall Description}
\subsection{Product perspective}
	\begin{enumerate}
		\item These products, the Eclipse Image Processor, Eclipse Image
		Processor Developer Pipeline, and the Eclipse Simulator, are all components of
		the larger Eclipse Megamovie Project. The Eclipse Image Processor will be
		a binary that processes images, finds the sun and moon in each image, and
        exports metdata about the images. The Eclipse Image Processor Developer
        Pipeline is an interface that allows easy experimentation and development
        with the image processor via a test script that builds and runs the processor, can
        download images, and uploads an HTML document to Google Cloud Storage
        with nicely formatted output. The Eclipse Simulator will be a standalone
        JavaScript module that can be added to an existing webpage.

		\item The Eclipse Image Processor and Image Processor Developer Pipeline
		do not directly interface with the user, the pipeline will function as a
        developer tool and the Image Processor will eventually function as a backend
        system for file uploads on the Eclipse Megamovie website.
		The Eclipse Simulator does interface
		directly with the user. It should function on most modern
		internet browsers (Chrome, Firefox, Safari, Edge). The simulator
		will appear to the user as a 2D animated depiction of the
		Sun and the Moon as they appear at the specified time and
		location. The simulator will also have background imagery in
		addition to the Sun and the Moon. Besides the images, the
		simulator will have a time slider, a location input, a zoom feature,
        and a play buttob.

		\item This system does not interface with hardware.

		\item The Eclipse Image Processor and Image Processor Developer Pipeline
		will be designed to run on Ubuntu 16.04. It is necessary for these
		applications to be compatible with this operating system because
		the machines that will be running the them also run Ubuntu 16.04. The Eclipse
		Simulator is a JavaScript module that will work on modern
		browsers like Chrome, Firefox and Safari. We expect our
		users will use these popular browsers so it is necessary
		for our product to interface with them.
	\end{enumerate}

\subsection{Product functions}
	\begin{enumerate}
		\item The Eclipse Simulator will be a standalone JavaScript
		module enabling users to "preview" the eclipse. It will be
		designed in a stylized, 2D manner. The simulator will
		incorporate a time slider that allows users to simulate the
		eclipse in a time window spanning from about 1.5 hours before maximum
        eclipse to about 1.5 hours. As users drag the time
		slider, the eclipse will animate in the simulator window.
		The view of the eclipse which users are presented will be
		specific to the selected location.

		\item The Eclipse Image Processor application will ingest eclipse
		photos and find the sun and moon in these images. In addition to
        identifying circles in the images the image processor will export
        data about its run to an output directory. This function will allow
        developers to imporove upon the application as it matures.

		\item The Eclipse Image Processor Developer Pipeline will enable easy
        experimentation and testing of the image processor. This includes
		collecting images from Google Cloud Storage for the Image Processor to process,
		invoking the Image Processor with these images as input, and collecting
		the output of the Image Processor and uploading it in a nicely formatted
        HTML document to Google Cloud Storage.

	\end{enumerate}

\subsection{User characteristics}
	\begin{enumerate}
		\item The Eclipse Image Processor application will be used by
            the members of this project.

		\item The Eclipse Image Processor Developer Pipeline application will be
            used by Google Engineers.

		\item The Eclipse Simulator application will be used by the
			general public. No unusual technical/scientific knowledge is
			expected of these users. It is assumed however, that these users
			are familiar with the internet and web browsers.
	\end{enumerate}

\subsection{Constraints}
None.

\subsection{Assumptions and dependencies}
	\begin{enumerate}
		\item This SRS assumes the availability of Ubuntu 16.04.

		\item This SRS assumes the availability of Google Cloud Platform
		n1-standard-4 virtual machines.

		\item This SRS assumes the availability of the OpenCV computer
		vision library.
	\end{enumerate}

\subsection{Apportioning of requirements}
See Gantt Chart in Appendix.

%%%%%%%%%%%%%%%%%%%%%%%%%%%%%%%%%%%
%  SECTION 3: REQUIREMENTS
%%%%%%%%%%%%%%%%%%%%%%%%%%%%%%%%%%%%%%%
\newpage
\section{Specific requirements}

% SECTION 3.1
\subsection{External Interfaces}

\subsubsection{Eclipse Simulator}
	\begin{enumerate}
		\item The simulator is a standalone JavaScript module that can
		be included on an existing webpage.

		\item Users can select the location from which to simulate the eclipse.
		This can be entered at any point while using the simulator.
		\begin{enumerate}
			\item Location can be entered as: latitude/longitude, address, zip code,
			city name, state name.
			\item Initial simulator location can be programmatically set as
			initialization parameter.
		\end{enumerate}

		\item Users will be able to adjust the simulator time from 12 before
		the eclipse to 12 hours after it.
		\begin{enumerate}
			\item Time can be advanced via a draggable slider or clickable buttons.
		\end{enumerate}

	\end{enumerate}

\subsubsection{Eclipse Image Processor}
	\begin{enumerate}
		\item The Image Processor will be compatible with Ubuntu 16.04 and
		 will include a script that to install all dependencies/build the
		 binary.

		 \item The application will accept the following input as command line
		 arguments:
		 \begin{enumerate}
		 	\item Required: image\_list\_file
		 		\begin{enumerate}
		 			\item Absolute or relative (to the directory the binary was
		 			invoked from) path to file containing a list of image filenames
		 			with no directory prefix.
		 		\end{enumerate}

		 	\item Required: output\_dir
		 		\begin{enumerate}
		 			\item Directory to write output files to.
		 		\end{enumerate}

		 	\item Optional: image\_path\_prefix
		 		\begin{enumerate}
		 			\item Absolute or relative (to the directory the binary was
		 			 invoked from) path to prepend to each image filename in
		 			 image\_file\_list. Defaults to "./".
		 		\end{enumerate}
		 \end{enumerate}

		 \item The application will accept JPEG (.jpeg/.jpg) and PNG (.png)
		 image files.
		 \begin{enumerate}
		 	\item Images of an invalid format will be disregarded and an error
		 	 message will be written to stderr.
		 \end{enumerate}

		 \item The application will write the following output to the
		 output\_dir directory:
		 \begin{enumerate}
		 	\item image\_transformations.txt
		 	\begin{enumerate}
		 		\item File containing one line per image processed with the
		 		following values (comma separated):
		 		\begin{enumerate}
		 			\item processed\_image: processed image filename

		 			\item crop\_topl\_x: x coordinate of top left corner of
		 			cropped image (refers to rotated image)

		 			\item crop\_topl\_y: y coordinate of top left corner of
		 			cropped image (refers to rotated image)

		 			\item crop\_botr\_x: x coordinate of bottom right corner of
		 			cropped image (refers to rotated image)

		 			\item crop\_botr\_y: y coordinate of bottom right corner of
		 			 cropped image (refers to rotated image)

		 			\item timestamp: timestamp at which image was taken

		 			\item eclipse\_path\_percent: percentage through eclipse
		 			totality path at which image was taken, see requirement \#8
		 			 of 3.2.2

		 			 \item rejected\_reason: reason a particular image was rejected,
		 			 see requirement \#10 of 3.2.2.
		 		\end{enumerate}
		 	\end{enumerate}
			\item Processed image files
				\begin{enumerate}
					 \item Processed image files will be saved into a sub-directory
					 of output\_dir called "images" and will be named as follows:
					 *\_pp.[png$\vert$jpeg$\vert$jpg].
				\end{enumerate}
		 \end{enumerate}

		 \item Images that are rejected will  be added to the
		 image\_transformations.txt file but will not be cropped/exported to
		 the images directory.

		 \item The application will format all log messages as follows:
		 \begin{enumerate}
		 	\item img\_preproc:level:timestamp:”message”
		 	\begin{enumerate}
		 		\item Values of level: ERROR/WARNING/INFO/DEBUG
		 		\item Value of timestamp: current timestamp
		 		\item Value of message: specific logging message
		 	\end{enumerate}
		 \end{enumerate}
	\end{enumerate}

\subsubsection{Eclipse Image Processor Manager}
	\begin{enumerate}
		\item The Eclipse Image Processor Manager will collect images to be processed
		from Google Cloud Storage. It will obtain lists of files to download and their
		respective metadata from Google Cloud Datastore. It will be aware of the scheme
		of the Datastore NoSQL database in order to read/write image metadata.

		\item The Eclipse Image Processor Manager will invoke instances of the Eclipse
		Image Processor and communicate with them via a command-line interface as well
		as data files. The format of these data files, as well as the command-line interface
		is described in 3.1.2.
	\end{enumerate}

%%%%%% Section 3.2 %%%%%%%%
\subsection{Functional Requirements}

\subsubsection{Eclipse Simulator}
	\begin{enumerate}
		\item Displayed solar/lunar placement will be based on location and
		time and will account for edge cases like when the location is not in
		the path of totality.

		\item The simulator location will be restricted to North America as this is where
		the eclipse will be visible from.

		\item The Simulator will display a timestamp indicating the time that the simulator
		is set to.
	\end{enumerate}

\subsubsection{Eclipse Image Processor}
	\begin{enumerate}
		\item Invalid JPEG and PNG files (e.g. cannot be opened by OpenCV,
		width/height equal to 0px, etc.) will be ignored and an error message
		will be written to stderr.

		\item Images where the solar disk has a radius of less than 50px
		will be rejected (see requirement \#5 in 3.1.2).

		\item The application will crop the images to be square with the
		Sun centered. The images will be cropped so that there is a 100px
		pad between the solar perimeter and the edge of the image on all
		sides.
		\begin{enumerate}
			\item Images that do not have enough room around the Sun to
			allow for the pad described above will be rejected, see requirement \#5 in 3.1.2.
		\end{enumerate}

		\item Based on images' GPS EXIF information, the application will
		compute/export the percentage through the eclipse's path of
		totality at which the image was taken. 0\% will be defined as the
		westmost point on that path of totality that is over land, this
		point is on the west coast of Oregon. 100\% will be defined as the
		eastmost point on the path of totality that is over land, this
		point is on the east coast of South Carolina. The application will
		compute the point on the path of totality nearest the point where
		the image was taken. This point will be used to compute the
		percentage through the path of totality at which the image was
		taken.

		\item Images  that were not taken on the path of
		totality will be rejected, see requirement \#5 in 3.1.2.

		\item Images can be rejected by the Image Processor. When an image is rejected,
		the reason it was rejected will be noted and exported, see requirement \#5 in 3.1.2.
		The possible reasons for which an image can be rejected will include at least the
		following:
		\begin{enumerate}
			\item Invalid image file, e.g. image file is corrupted and cannot be opened.

			\item Solar disk radius is too small.

			\item There is not enough padding around the solar disk.

			\item Image was not taken on the path of totality.
		\end{enumerate}

	\end{enumerate}

\subsubsection{Eclipse Image Processor Manager}
	\begin{enumerate}
		\item The application will be designed so that multiple instances of it
		can run concurrently with no communication between them.
		These distinct managers will not attempt to process the same images.

		\item The application will record the time at which it "checks-out" image
		files for processing in Datastore. This enables future processing of images
		that were downloaded by an Image Processor Manager instance that later became
		delinquent.

		\item The application will group files/metadata downloaded from Google
		Cloud into directories corresponding to individual workloads that will be
		delegated to distinct Image Processor instances.

		\item When an Image Processor instance completes its workload, its output data
		will be collected by its parent Image Processor Manager and uploaded to Google
		Cloud. This will include processed image files and metadata that is saved to
		Datastore.

		\item The application will kill delinquent Image Processor Instances and upload
		the results of processing any images that were processed successfully. Images that
		were not processed successfully will be marked as such in Datastore.

		\item The application will be highly parallel. Multiple Eclipse Image Processor
		application instances will be launched concurrently, and while these are working,
		the next images to be processed will be downloaded and previous results will be
		uploaded.
		\begin{enumerate}
			\item The number of Image Processor instances launched will be determined by
			the number of cores on the host VM. \textit{Note: this does not mean that
			given an \(n\) core machine, \(n\) Image Processor instances will be launched - some cores
			must be reserved for downloading/preparing input data, and processing/uploading results}.
		\end{enumerate}

	\end{enumerate}

% SECTION 3.3 %%%%%%%%%%%
\subsection{Performance Requirements}

\subsubsection{Eclipse Simulator}
	\begin{enumerate}
		\item All simulator resources will load in less than 500ms given a 1-10
		 Mbps internet connection.
	\end{enumerate}

\subsubsection{Eclipse Image Processor}
	\begin{enumerate}
		\item The application should take less than 5 seconds to process an
		 image when running on a Google Cloud Platform n1-standard-4 virtual
		 machine.
	\end{enumerate}

\subsubsection{Eclipse Image Processor Manager}
	\begin{enumerate}
		\item The Image Processor Manager will saturate either the CPUs or network interface of
		the host VM to which it is deployed.
		\begin{itemize}
			\item The reason for the "or" in the above statement is that either the time to process
			images, or the time to download them and upload their processed counterparts will be a
			bottleneck depending on the VM configuration. The goal here is to get as close to network
			saturation and 100\% utilization of all CPU cores as possible. If all CPU cores are running
			at nearly 100\% but the network interface is not saturated, then VMs with more cores should
			be used. Conversely, if the network interface is saturated but not all cores are running at
			100\%, VMs with fewer cores should be used.
		\end{itemize}
	\end{enumerate}

%%%%%%%%%%%%%%%%%%%%%%%%%%%%%%%%%
%  SECTION 4: Supporting Information
%%%%%%%%%%%%%%%%%%%%%%%%%%%%%%%%%%
\newpage
\section{Supporting Information}

\subsection{Appendix}

\begin{enumerate}
	\item Figure 1. Project Gantt Chart
\end{enumerate}

\begin{center}
	\includegraphics[width=\textwidth]{gantt1.eps}
\end{center}

\end{document}

