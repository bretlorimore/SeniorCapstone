\documentclass[10pt, onecolumn, draftclsnofoot, letterpaper, compsoc]{IEEEtran}

\usepackage{cite}
\usepackage{hyperref}
\usepackage{enumitem}

%%%%%%%%%%%%%%%%%%%%%%%%%%%%%%%%%%%%%%%%%%%%%%%%%%%%%
% Macro for the signatures at the end                %
\newcommand*{\SignatureAndDate}[1]{%
    \par\noindent\makebox[2.5in]{\hrulefill} \hfill\makebox[2.0in]{\hrulefill}%
    \par\noindent\makebox[2.5in][l]{#1}      \hfill\makebox[2.0in][l]{Date}%
}%

\renewcommand*\contentsname{Table of Contents} % Rename ToC

\newcommand{\myindent}{\hspace{\oldparindent}}

% Temp title and author
\title{Requirements}
\date{\today} % Somehow this isn't working..
\author{Totality AweSun \\
		Bret~Lorimore, Jacob~Fenger, George~Harder \\
		\textit{November 4th, 2016 \\
		CS 461 - Fall 2016}} 

\begin{document}

\setlist[itemize]{topsep=1pt} % EDIT LISTS

\maketitle

\begin{abstract}

On August 21, 2017 a total solar eclipse will pass over the United States. The
path of totality will stretch from Oregon to South Carolina. There has not been
a total solar eclipse like this, crossing the country from coast to coast, since
the eclipse of 1918. The Eclipse Megamovie Project is a collaboration between
Google and scientists from UC Berkeley and several other institutions with the
aim of compiling a large dataset of eclipse observations. Acquiring coronal data
is of particular interest as the corona is not normally visible from Earth.
Specifically, the project will crowdsource photos of the eclipse from
photographers at various locations along the path of totality. These images will
be aligned spatially and temporally and stitched into a unique movie that shows
the eclipse over a period of 1.5 hours as it passes across the United States.
Additionally, the complete photo dataset will be open sourced so that
independent researchers may do their own analysis.

Google will contribute applications providing, among other things, backend image
processing, photo upload capabilities, and static informational content. This
senior capstone project will consist of two distinct sub-projects, specifically,
improving/implementing an image processing algorithm facilitating the
classification and spatial/temporal alignment of solar eclipse images before
they are stitched into a movie, and a location-based eclipse simulator.

\end{abstract}


\newpage

\tableofcontents

\newpage
%%%%%%%%%%%%%%%%%%%%%%%%%%%%%%%%%
%  SECTION 1: INTRODUCTION
%%%%%%%%%%%%%%%%%%%%%%%%%%%%%%%%%%
\section{Introduction}

\subsection{Purpose}
The purpose of this software requirements specification (SRS) is to describe in
detail the Eclipse Image Processor and Eclipse Simulator that our group will 
produce. By writing these requirements down and agreeing to them with our 
sponsor both parties will have a clear understanding of what the finished 
product will be able to do. The intended audience for this SRS is our sponsor, 
the Senior Capstone Instruction Team, and ourselves.

\subsection{Scope}
We are producing two products: an Eclipse Image Processor and an Eclipse 
Simulator. The Eclipse Image Processor will take images, find the sun and the
moon in the images, crop the image around these bodies, and sort the images 
based on the relative position of the Sun and the Moon. The Eclipse Simulator 
will provide users with a 2-D visual representation of the eclipse from a 
specified location from 12 hours before it occurs to 12 hours after.

\subsection{Definitions, acronyms, and abbreviations}
	
	\textbf{Eclipse Megamovie Project:}
	The Eclipse Megamovie Project is a collaboration between Google 
	and scientists from Berkeley and several other institutions with the 
	aim of collecting large quantities of observations of the solar eclipse
	that will pass over the United States on August 21, 2017. The project
	will crowdsource photos of the eclipse from photographers at various 
	points along the path of totality. \\

	\noindent \textbf{EXIF:}
	EXIF refers to the Exchangeable Image File Format, a standard 
	media file format. Ancillary data tags associated with other media 
	files are frequently referred to as "EXIF fields", "EXIF data", etc.
	This is how EXIF will be used within the scope of this document. \\

	\noindent \textbf{GPS:}
	GPS refers to a global positioning system. We will refer to the
	latitude and longitude information obtained from a global positioning 
	system as "GPS coordinates", "GPS data", etc. throughout this document. \\

	\noindent \textbf{JPEG/JPG:}
	JPEG is a lossy compression technique for images. When we refer
	to JPEG/JPG files in this document we are referring to image files 
	compressed in this method with the .jpeg or .jpg file extension. \\

	\noindent \textbf{PNG:}
	PNG refers to the Portable Network Graphics image file format. 
	Images in the PNG format are frequently referred to as "PNGs" and are 
	saved with the .png file extension. 


\subsection{References}

This SRS makes reference to a report by Larisza D. Krista and Scott W.  
McIntosh titled "The Standardisation and Sequencing of Solar Eclipse Images for 
the Eclipse Megamovie Project." This technical report was produced by a 
collaboration between scientists at the University of Colorado at Boulder, the
 National Center for Atmospheric Research and the National Oceanic and 
 Atmospheric Administration. This paper can be found on arxiv.org.

\subsection{Overview}

The remainder of this SRS contains an overall description of the Eclipse Image
 Processor and Eclipse Simulator systems in section 2. Following these 
 descriptions are specific requirements for the systems in section 3.

%%%%%%%%%%%%%%%%%%%%%%%%%%%%%%%%%%%
%  SECTION 2: OVERALL DESCRIPTION
%%%%%%%%%%%%%%%%%%%%%%%%%%%%%%%%%%%%%%%
\section{Overall Description}
\subsection{Product perspective}
	\begin{enumerate}
		\item These products, the Eclipse Image Processor and the 
		Eclipse Simulator, are both components of the larger Eclipse 
		Megamovie Project. They are designed to operate as wholly 
		independent modules that can be ‘plugged into’ the existing 
		Eclipse Megamovie codebase. The Eclipse Image Processor will be
		a binary that receives images from another application, 
		processes them, and exports them. The Eclipse Simulator will 
		be a standalone JavaScript module that can be added to an 
		existing webpage.

		\item The Eclipse Image Processor does not directly interface
		with the user. It resides behind the front-end of the Eclipse
		Megamovie website. The Eclipse Simulator does interface 
		directly with the user. It should function on most modern 
		Internet browsers (Chrome, Firefox, Safari). The simulator 
		will appear to the user as a 2-D animated depiction of the 
		Sun and the Moon as they appear at the specified time and 
		location. The simulator will also have background imagery in
		addition to the sun and the moon. Besides the images, the
		simulator will have a time slider, a location input, and a
		time display.

		\item This system does not interface with hardware.

		\item The Eclipse Image Processor will be designed to work on
		Ubuntu 16.04. It is necessary for the image processor to 
		work on this operating system because the machines that will
		be running the processor use Ubuntu 16.04. The Eclipse 
		Simulator is a JavaScript module that will work on modern 
		browsers like Chrome, Firefox and Safari. We expect our 
		users will use these popular browsers so it is necessary 
		for our product to interface with them.      
	\end{enumerate}

\subsection{Product functions}
	\begin{enumerate}
		\item Eclipse Image Processor
		\begin{enumerate}
			\item The Eclipse Image Processor application will ingest 
			photos of the eclipse, align them spatially, and categorize
			them to help recover temporal ordering. These aligned and 
			categorized images will be cropped so that the sun 
			occupies the same amount of each image and will be 
			exported. All data that is required to take the raw input 
			images and produce the exported images will be saved to a 
			data file. Additionally, select EXIF information will be 
			extracted from the image files and included in this output
			data file.
		\end{enumerate}

		\item Eclipse Simulator
		\begin{enumerate}
			\item The Eclipse Simulator will be a standalone JavaScript 
			module enabling users to preview the eclipse. It will be 
			designed in a stylized, 2D manner. The simulator will 
			incorporate a time slider that allows users to simulate the 
			eclipse in a time window spanning from 12 hours before the 
			eclipse to 12 hours after it. As users drag the time 
			slider, the eclipse will animate in the simulator window. 
			The view of the eclipse which users are presented will be 
			specific to the selected location.
		\end{enumerate}
	\end{enumerate}

\subsection{User characteristics}
	\begin{enumerate}
		\item Eclipse Image Processor
		\begin{enumerate}
			\item The Eclipse Image Processor application will be used by 
			Google Engineers.
		\end{enumerate}

		\item Eclipse Simulator
		\begin{enumerate}
			\item The Eclipse Simulator application will be used by the
			general public. No unusual technical/scientific knowledge is
			expected of these users. These users are presumed to be
			familiar with the internet and web browsers.
		\end{enumerate}
	\end{enumerate}

\subsection{Constraints}
None.

\subsection{Assumptions and dependencies}
	\begin{enumerate}
		\item This SRS assumes the availability of Ubuntu 16.04.
		\item This SRS assumes the availability of Google Cloud Platform n1-
		standard-4 virtual machines.
		\item This SRS assumes the availability of the OpenCV computer 
		vision library.
	\end{enumerate}

\subsection{Apportioning of requirements}
See Gantt Chart in Appendix
	

%%%%%%%%%%%%%%%%%%%%%%%%%%%%%%%%%%%
%  SECTION 3: REQUIREMENTS
%%%%%%%%%%%%%%%%%%%%%%%%%%%%%%%%%%%%%%%
\section{Specific requirements}

% SECTION 3.1
\subsection{External Interfaces}

\subsubsection{Eclipse Simulator}
	\begin{enumerate}
		\item The simulator is a standalone JavaScript module that can easily 
		be included on an existing webpage.

		\item Users can select the location from which to simulate the eclipse. 
		This can be entered at any point while using the simulator.
		\begin{enumerate}
			\item Location can be entered as: lat/long, address, zip code, city 
			name, state name.
			\item Initial simulator location can be programmatically set as 
			initialization parameter.
		\end{enumerate}

		\item Users will be able to adjust the simulator time from 12 before 
		the eclipse to 12 hours after it.
		\begin{enumerate}
			\item Time can be advanced via a slider, or clickable buttons.
		\end{enumerate}

	\end{enumerate}

\subsubsection{Eclipse Image Processor}
	\begin{enumerate}
		\item The image pre-processor will be compatible with Ubuntu 16.04 and
		 will include a script that to install all dependencies/build the 
		 binary.

		 \item The application will accept the following input as command line
		 arguments:
		 \begin{enumerate}
		 	\item Required: image\_list\_file
		 		\begin{enumerate}
		 			\item Absolute or relative (to the directory the binary was 
		 			invoked from) path to file containing a list of image file 
		 			names with no directory.
		 		\end{enumerate}

		 	\item Required: output\_dir
		 		\begin{enumerate}
		 			\item Directory to write output files to.
		 		\end{enumerate}

		 	\item Optional: image\_path\_prefix
		 		\begin{enumerate}
		 			\item Absolute or relative (to the directory the binary was
		 			 invoked from) path to prepend to each image filename in 
		 			 image\_file\_list. Defaults to "./".
		 		\end{enumerate}
		 \end{enumerate}

		 \item The application will accept JPEG (.jpeg/.jpg) and PNG (.png) 
		 image files.
		 \begin{enumerate}
		 	\item Images of an invalid format will be disregarded and an error
		 	 message will be written to stderr.
		 \end{enumerate}

		 \item The application will write the following output to the 
		 output\_dir directory:
		 \begin{enumerate}
		 	\item image\_transformations.txt
		 	\begin{enumerate}
		 		\item File containing one line per image processed with the 
		 		following values (comma separated):
		 		\begin{enumerate}
		 			\item processed\_image: processed image filename

		 			\item image\_type: image type (FULL\_DISK/TOTALITY/etc.)

		 			\item rot\_angle: angle original image was rotated (degrees)

		 			\item crop\_topl\_x: x coordinate of top left corner of 
		 			cropped image (refers to rotated image)

		 			\item crop\_topl\_y: y coordinate of top left corner of 
		 			cropped image (refers to rotated image)

		 			\item crop\_botr\_x: x coordinate of bottom right corner of 
		 			cropped image (refers to rotated image)

		 			\item crop\_botr\_y: y coordinate of bottom right corner of
		 			 cropped image (refers to rotated image)

		 			\item rel\_center\_offset: relative offset of solar/lunar
		 			 disk centers, see requirement \#3 of 3.2.2  (optional, only included
		 			  for CRESCENT type images)

		 			\item diamond\_rel\_size: size of “diamond” relative to the 
		 			size of the solar disk, see requirement \#4 of 3.2.2 (
		 			optional, only for DIAMOND\_RING type images)

		 			\item timestamp: timestamp at which image was taken (
		 			optional, only included when images have an EXIF timestamp 
		 			field)

		 			\item eclipse\_path\_percent: percentage through eclipse 
		 			totality path at which image was taken, see requirement \#8
		 			 of 3.2.2 (optional, only included for images with GPS 
		 			 coordinate EXIFfield)
		 		\end{enumerate}
		 	\end{enumerate}
			\item Pre-processed image files
				\begin{enumerate}
					 \item images/*\_pp.[png$\vert$jpeg$\vert$jpg]
				\end{enumerate}
		 \end{enumerate}

		 \item Images that are rejected will not be added to the 
		 image\_transformations.txt file and will not be cropped/exported to
		  the images directory. When an image is rejected a message with this 
		 information will be printed to stdout.

		 \item The application will format all log messages as follows:
		 \begin{enumerate}
		 	\item img\_preproc:level:timestamp:”message”
		 	\begin{enumerate}
		 		\item Values of level: ERROR/WARNING/INFO/DEBUG
		 		\item Value of timestamp: current timestamp
		 		\item Value of message: specific logging message
		 	\end{enumerate}
		 \end{enumerate}
	\end{enumerate}

%%%%%% Section 3.2 %%%%%%%%
\subsection{Functional Requirements}

\subsubsection{Eclipse Simulator}
	\begin{enumerate}
		\item Displayed solar/lunar placement will be based on location and 
		time and will account for edge cases like when the location is not in 
		the path of totality. For example, if the location is on the opposite 
		side of the world as the eclipse, the simulator should shift to a night
		 time display.

		\item Simulator will display the local time that the simulator is set 
		to, e.g. there is a well defined time associated with the user 
		selecting Corvallis, Oregon as their location and a simulator time of 
		-3:13 (3 hours 13 minutes) before the eclipse. This time should be 
		displayed on the simulator.
	\end{enumerate}

\subsubsection{Eclipse Image Processor}
	\begin{enumerate}
		\item Invalid JPEG and PNG files (e.g. cannot be opened by OpenCV, width
		/height equal to 0px, etc.) will be ignored and an error message will
		 be written to stderr.

		\item The application will classify the input images as being one of
		the following types:
		\begin{enumerate}
			\item FULL\_DISK
			\begin{enumerate}
				\item Image of an unobscured solar disk.
			\end{enumerate}

			\item TOTALITY
			\begin{enumerate}
				\item Image of a total solar eclipse.
			\end{enumerate}

			\item CRESCENT
			\begin{enumerate}
				\item Image of a partially eclipsed sun, creating a "crescent" 
				shape.
			\end{enumerate}

			\item DIAMOND\_RING
			\begin{enumerate}
				\item Image of a nearly fully eclipsed sun where there is one 
				"hot spot" on the sun’s perimeter. This hot spot along with the 
				sun’s perimeter have the shape of a diamond ring.
			\end{enumerate}
		\end{enumerate}

		\item For images of type CRESCENT, the application will compute/
		export a delta of the position of the center of the sun and the 
		moon relative to the size of the solar disk. This delta will be a 
		signed value based on the sun’s position, i.e. if the moon is to 
		the left of the sun (in the cropped/rotated image) the delta will 
		be negative and conversely, if the moon is to the right of the sun 
		the delta value will be positive.

		\item For images of type DIAMOND\_RING, the application will 
		compute/export the size of the "diamond" relative to the size of 
		the solar disk.

		\item Images where the solar disk has a radius of less than 50px
		will be rejected (see requirement \#5 in 3.1.2).

		\item The application will crop the images to be square with the 
		sun centered. The images will be cropped so that there is a 100px
		pad between the solar perimeter and the edge of the image on all 
		sides.
		\begin{enumerate}
			\item Images that do not have enough room around the sun to 
			allow for the pad described above will be rejected, see requirement \#5 in 3.1.2.
		\end{enumerate}

		\item The application will rotate DIAMOND\_RING and CRESCENT type 
		images so that they are aligned horizontally, as described by 
		Krista et al.

		\item For images with GPS EXIF information, the application will 
		compute/export the percentage through the eclipse’s path of 
		totality at which the image was taken. 0\% will be defined as the 
		westmost point on that path of totality that is over land, this 
		point is on the west coast of Oregon. 100\% will be defined as the 
		eastmost point on the path of totality that is over land, this 
		point is on the east coast of South Carolina. The application will
		compute the point on the path of totality nearest the point where 
		the image was taken. This point will be used to compute the 
		percentage through the path of totality at which the image was 
		taken.

		\item Images with GPS EXIF information that are not on the path of
		totality will be rejected, see requirement \#5 in 3.1.2.

	\end{enumerate}

% SECTION 3.3 %%%%%%%%%%%
\subsection{Performance Requirements}

\subsubsection{Eclipse Simulator}
	\begin{enumerate}
		\item All simulator resources will load in less than 500ms given a 1-10
		 Mbps internet connection. Once this is achieved file sizes will be 
		 optimized to achieve the fastest load time possible.
	\end{enumerate}

\subsubsection{Eclipse Image Processor}
	\begin{enumerate}
		\item The application should take less than 5 seconds to process an
		 image when running on a Google Cloud Platform n1-standard-4 virtual
		 machine.
	\end{enumerate}

\appendix

\vspace{20mm}
\noindent \SignatureAndDate{David Konerding, Project Sponsor}
\vspace{8mm}
\noindent \SignatureAndDate{Bret Lorimore}
\vspace{8mm}
\noindent \SignatureAndDate{George Harder}
\vspace{8mm}
\noindent \SignatureAndDate{Jacob Fenger}

\end{document}
