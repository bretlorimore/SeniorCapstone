\documentclass[10pt, onecolumn, draftclsnofoot, letterpaper, compsoc]{IEEEtran}

\usepackage{cite}
\usepackage{hyperref}
\usepackage[normalem]{ulem}
\usepackage{enumitem}
\usepackage{graphicx}

\graphicspath{ {images/} }

% Colored Text
\usepackage[usenames, dvipsnames]{color}

%%%%%%%%%%%%%%%%%%%%%%%%%%%%%%%%%%%%%%%%%%%%%%%%%%%%%
% Macro for the signatures at the end                %
\newcommand*{\SignatureAndDate}[1]{%
    \par\noindent\makebox[2.5in]{\hrulefill} \hfill\makebox[2.0in]{\hrulefill}%
    \par\noindent\makebox[2.5in][l]{#1}      \hfill\makebox[2.0in][l]{Date}%
}%

\renewcommand*\contentsname{Table of Contents} % Rename ToC

\newcommand{\myindent}{\hspace{\oldparindent}}

% Temp title and author
\title{Requirements}
\date{\today} % Somehow this isn't working..
\author{Totality AweSun \\
		Bret~Lorimore, Jacob~Fenger, George~Harder \\
		\textit{\today \\
		Senior Capstone, Oregon State University}}

\begin{document}

\setlist[itemize]{topsep=1pt} % EDIT LISTS

\maketitle

\begin{abstract}

On August 21, 2017 a total solar eclipse will pass over the United States. The
path of totality will stretch from Oregon to South Carolina. There has not been
a total solar eclipse like this, crossing the country from coast to coast, since
the eclipse of 1918. The Eclipse Megamovie Project is a collaboration between
Google and scientists from UC Berkeley and several other institutions with the
aim of compiling a large dataset of eclipse observations. Acquiring coronal data
is of particular interest as the corona is not normally visible from Earth.
Specifically, the project will crowdsource photos of the eclipse from
photographers at various locations along the path of totality. These images will
be aligned spatially and temporally and stitched into a unique movie that shows
the eclipse over a period of 1.5 hours as it passes across the United States.
Additionally, the complete photo dataset will be open sourced so that
independent researchers may do their own analysis.

Google will contribute applications providing, among other things, backend image
processing, photo upload capabilities, and static informational content. This
senior capstone project will consist of three distinct sub-projects, specifically,
implementing/modifying an image processing algorithm facilitating the
classification and alignment of solar eclipse images before they are stitched into
a movie, a manager to run this image processing app, and a location-based eclipse simulator.

\end{abstract}

\vspace{10mm}
\noindent \SignatureAndDate{David Konerding, Project Sponsor}
\vspace{8mm}
\noindent \SignatureAndDate{Bret Lorimore}
\vspace{8mm}
\noindent \SignatureAndDate{George Harder}
\vspace{8mm}
\noindent \SignatureAndDate{Jacob Fenger}

\newpage

\tableofcontents

\newpage
%%%%%%%%%%%%%%%%%%%%%%%%%%%%%%%%%
%  SECTION 1: INTRODUCTION
%%%%%%%%%%%%%%%%%%%%%%%%%%%%%%%%%%
\section{Introduction}

\subsection{Purpose}
The purpose of this software requirements specification (SRS) is to describe in
detail the Eclipse Image Processor, Eclipse Image Processor Manager, and the Eclipse
Simulator that our group will
produce. By writing these requirements down and agreeing to them with our
sponsor both parties will have a clear understanding of what the finished
product will be and what it will be able to do. The intended audience for this
SRS is our sponsor, the Senior Capstone Instruction Team, and ourselves.

\subsection{Scope}
We are producing  three products: an Eclipse Image Processor, Eclipse
Image Processor Manager, and an Eclipse
Simulator. The Eclipse Image Processor will ingest images and if
they are of a total solar eclipse, align/crop these images consistantly. Images of non-total
solar eclipses will marked as such and otherwise ignored. The Eclipse Image Processor Manager
will download images to be processed by the Image Processor, run the Image Processor on these images,
and collect the results of running the Image Processor and upload them to Google Cloud. The Eclipse
Simulator will provide users with a 2D visual representation of the eclipse from a
specified location from 12 hours before it occurs to 12 hours after.

\subsection{Definitions, acronyms, and abbreviations}

	\textbf{Eclipse Megamovie Project:}
	The Eclipse Megamovie Project is a collaboration between Google
	and scientists from Berkeley and several other institutions with the
	aim of collecting large quantities of observations of the solar eclipse
	that will pass over the United States on August 21, 2017. The project
	will crowdsource photos of the eclipse from photographers at various
	points along the path of totality. \\

	\noindent \textbf{EXIF:}
	EXIF refers to the Exchangeable Image File Format, a standard
	media file format. Ancillary data tags associated with other media
	files are frequently referred to as "EXIF fields", "EXIF data", etc.
	This is how "EXIF" will be used within the scope of this document. \\

	\noindent \textbf{GPS:}
	GPS stands for global positioning system. We will refer to the
	latitude and longitude information obtained from a global positioning
	system as "GPS coordinates", "GPS data", etc. throughout this document. \\

	\noindent \textbf{JPEG/JPG:}
	JPEG is a lossy compression technique for images. When we refer
	to JPEG/JPG files in this document we are referring to image files
	compressed in this method with the .jpeg or .jpg file extension. \\

	\noindent \textbf{PNG:}
	PNG refers to the Portable Network Graphics image file format.
	Images in the PNG format are frequently referred to as "PNGs" and are
	saved with the .png file extension.


\subsection{References}

This SRS makes reference to a report by Larisza D. Krista and Scott W.
McIntosh titled "The Standardisation and Sequencing of Solar Eclipse Images for
the Eclipse Megamovie Project." This technical report was produced as a
collaboration between scientists at the University of Colorado at Boulder, the
National Center for Atmospheric Research and the National Oceanic and
Atmospheric Administration. This paper can be found on arxiv.org.

\subsection{Overview}

The remainder of this SRS contains an overall description of the Eclipse Image
Processor, Eclipse Image Processor Manager, and Eclipse Simulator systems in
section 2. Following these descriptions are specific requirements for the
systems in section 3.

%%%%%%%%%%%%%%%%%%%%%%%%%%%%%%%%%%%
%  SECTION 2: OVERALL DESCRIPTION
%%%%%%%%%%%%%%%%%%%%%%%%%%%%%%%%%%%%%%%
\newpage
\section{Overall Description}
\subsection{Product perspective}
	\begin{enumerate}
		\item These products, the Eclipse Image Processor, Eclipse Image
		Processor Manager, and the Eclipse Simulator, are all components of
		the larger Eclipse Megamovie Project. The Eclipse Image Processor will be
		a binary that receives images from another application - the Eclipse
		Image Processor Manager - processes them, and exports them. The Eclipse
		Image Processor and its manager will work together and form a standalone module
		that can be deployed to multiple Virtual Machines as a Docker container.
		The Eclipse Simulator will be a standalone JavaScript module that can be added to an
		existing webpage.

		\item The Eclipse Image Processor and Image Processor Manager
		do not directly interface with the user, they are backend
		systems that will be running asynchronously in Google Cloud.
		The Eclipse Simulator does interface
		directly with the user. It should function on most modern
		internet browsers (Chrome, Firefox, Safari). The simulator
		will appear to the user as a 2D animated depiction of the
		Sun and the Moon as they appear at the specified time and
		location. The simulator will also have background imagery in
		addition to the Sun and the Moon. Besides the images, the
		simulator will have a time slider, a location input, and a
		time display.

		\item This system does not interface with hardware.

		\item The Eclipse Image Processor and Image Processor Manager
		will be designed to run on Ubuntu 16.04. It is necessary for these
		applications to be compatible with this operating system because
		the machines that will be running the them also run Ubuntu 16.04. The Eclipse
		Simulator is a JavaScript module that will work on modern
		browsers like Chrome, Firefox and Safari. We expect our
		users will use these popular browsers so it is necessary
		for our product to interface with them.
	\end{enumerate}

\subsection{Product functions}
	\begin{enumerate}
		\item The Eclipse Simulator will be a standalone JavaScript
		module enabling users to "preview" the eclipse. It will be
		designed in a stylized, 2D manner. The simulator will
		incorporate a time slider that allows users to simulate the
		eclipse in a time window spanning from 12 hours before the
		eclipse to 12 hours after it. As users drag the time
		slider, the eclipse will animate in the simulator window.
		The view of the eclipse which users are presented will be
		specific to the selected location.

		\item The Eclipse Image Processor application will ingest
		photos of the eclipse, determine if they are images of a
		total solar eclipse, and if so, align them spatially.
		These aligned
		images will be cropped so that the Sun
		occupies the same amount of space in each image and will be
		exported. All data that is required to take the raw input
		images and produce the exported images will be saved to a
		data file. Additionally, select EXIF information will be
		extracted from the image files and included in this output
		data file. Images that are deemed as being of "non-total" solar
		eclipses will be marked as such and otherwise ignored.

		\item The Eclipse Image Processor Manager will be responsible for
		managing the Eclipse Image Processor application. This includes
		collecting images from Google Cloud for the Image Processor to process,
		invoking the Image Processor with these images as input, and collecting
		the output of the Image Processor and uploading it to Google Cloud. The
		Image Processor and Image Processor Manager will be deployed together
		in a single docker container to Google Container Engine Clusters (of VMs).
		An important role of the Image Processor Manager is that it will be responsible
		for ensuring that the compute resources on the host VMs are as saturated as
		possible. This means invoking multiple Image Processor processes concurrently,
		while at the same time downloading the next images to be processed and uploading
		the already processed images.


	\end{enumerate}

\subsection{User characteristics}
	\begin{enumerate}
		\item The Eclipse Image Processor application will be used by
			 the members of this project.

		\item The Eclipse Image Processor Manager application will be used by
			Google Engineers.

		\item The Eclipse Simulator application will be used by the
			general public. No unusual technical/scientific knowledge is
			expected of these users. It is assumed however, that these users
			are familiar with the internet and web browsers.
	\end{enumerate}

\subsection{Constraints}
None.

\subsection{Assumptions and dependencies}
	\begin{enumerate}
		\item This SRS assumes the availability of Ubuntu 16.04.

		\item This SRS assumes the availability of Google Cloud Platform
		n1-standard-4 virtual machines.

		\item This SRS assumes the ability to interface with Google Cloud from
		Python, for example with the Google Cloud Client Library for Python.

		\item This SRS assumes the availability of the OpenCV computer
		vision library.
	\end{enumerate}

\subsection{Apportioning of requirements}
See Gantt Chart in Appendix.

%%%%%%%%%%%%%%%%%%%%%%%%%%%%%%%%%%%
%  SECTION 3: REQUIREMENTS
%%%%%%%%%%%%%%%%%%%%%%%%%%%%%%%%%%%%%%%
\newpage
\section{Specific requirements}

% SECTION 3.1
\subsection{External Interfaces}

\subsubsection{Eclipse Simulator}
	\begin{enumerate}
		\item The simulator is a standalone JavaScript module that can
		be included on an existing webpage.

		\item Users can select the location from which to simulate the eclipse.
		This can be entered at any point while using the simulator.
		\begin{enumerate}
			\item Location can be entered as: latitude/longitude, address, zip code,
			city name, state name.
			\item Initial simulator location can be programmatically set as
			initialization parameter.
		\end{enumerate}

		\item Users will be able to adjust the simulator time from 2 before
		the eclipse to 2 hours after it.
		\begin{enumerate}
			\item Time can be advanced via a draggable slider or clickable buttons.
		\end{enumerate}

        \item Users will be able to "play" the eclipse and advance through the available time
        domain automatically by pressing a play button.

	\end{enumerate}

\subsubsection{Eclipse Image Processor}
	\begin{enumerate}
		\item The Image Processor will be compatible with Ubuntu 16.04 and
		 will include instructions to install all dependencies and a makefile to
         build the binary.

		 \item The application will accept the following input as command line
		 arguments:
		 \begin{enumerate}
		 	\item Required: images\_file
		 		\begin{enumerate}
		 			\item Absolute or relative (to the directory the binary was
		 			invoked from) path to file containing a list of image files to
                    process. Each line of this file should correspond to one image file.
                    Paths to image files must be absolute paths.
		 		\end{enumerate}

            \item Optional: mode
                \begin{enumerate}
                    \item Mode in which to run the image processor pipeline. Valid modes: "window", "batch"
                \end{enumerate}

		 	\item Optional: output\_dir
		 		\begin{enumerate}
		 			\item  Required in batch mode. Directory in which to save exported images and metadata when run in batch mode.
		 		\end{enumerate}

		 	\item Optional: hough\_dp
		 		\begin{enumerate}
		 			\item dp parameter to cv::HoughCircles.
		 		\end{enumerate}

            \item Optional: hough\_param1
		 		\begin{enumerate}
		 			\item param1 parameter to cv::HoughCircles.
		 		\end{enumerate}

            \item Optional: hough\_param2
		 		\begin{enumerate}
		 			\item param2 parameter to cv::HoughCircles.
		 		\end{enumerate}

            \item Optional: hough\_min\_dist
		 		\begin{enumerate}
		 			\item min\_dist parameter to cv::HoughCircles.
		 		\end{enumerate}

            \item Optional: hough\_min\_radius
		 		\begin{enumerate}
		 			\item min\_radius parameter to cv::HoughCircles.
		 		\end{enumerate}

            \item Optional: hough\_max\_radius
		 		\begin{enumerate}
		 			\item max\_radius parameter to cv::HoughCircles.
		 		\end{enumerate}
		 \end{enumerate}

		 \item The application will accept JPEG (.jpeg/.jpg) and PNG (.png)
		 image files.

		 \item The application will write the following output to the
		 output\_dir directory:
		 \begin{enumerate}
		 	\item metadata.txt
		 	\begin{enumerate}
		 		\item File containing one line per image processed with the
		 		following values ( |  separated):
		 		\begin{enumerate}
                    \item processed\_image: processed image filepath (absolute)

                    \item found\_circle(s): circles found by the image processor, format:
                        c(cente\_x, center\_y, radius)

                    \item execution\_time(s): time(s) taken for various parts of the image
                    processor to execute, format t(''name'', num\_secs)

                    \item observation(s): observations about the image, format: ''observation text''

                \end{enumerate}
		 	\end{enumerate}

            \item Processed image files
				\begin{enumerate}
					 \item Processed image files will be saved into output\_dir.
				\end{enumerate}
		 \end{enumerate}

	\end{enumerate}

\subsubsection{Eclipse Image Processor Developer Pipeline}
	\begin{enumerate}
		\item The developer pipeline will accept the following command line arguments:
            \begin{enumerate}
                \item Required: \$DIR, the directory to use for image/data storage. The following will be saved into DIR:
                \begin{enumerate}
                    \item If download flag set: A clone of the \$GCS\_BUCKET
                    \item A directory called output that will contain:
                    \begin{enumerate}
                        \item All the processed images \$GCS\_BUCKET
                        \item A metadata file that will contain the processed image names along with output information from the image processor
                        \item An HTML file that includes summarizes the image processor output info.
                    \end{enumerate}
                \end{enumerate}
            \end{enumerate}

        \item Required: \$GCS\_BUCKET, the Google Cloud Storage Bucket that contains the images to process.

        \item Optional: download, if set, the developer pipeline will download the images from Google Cloud Storage.
        Otherwise, it will assume these images are included in \$DIR/\$GCS\_BUCKET

        \item Optional: \$PIPELINE\_FLAGS, arguments to pass to the image processor when it is invoked.

	\end{enumerate}

%%%%%% Section 3.2 %%%%%%%%
\subsection{Functional Requirements}

\subsubsection{Eclipse Simulator}
	\begin{enumerate}
		\item Displayed solar/lunar placement will be based on location and
		time and will account for edge cases like when the location is not in
		the path of totality.

		\item The simulator location will be restricted to the United States.

		\item The simulator environment will darken as the eclipse progresses.

        \item The simulator will feature a zoom mode, where the sun appears larger
        in the sky. In zoom mode, the sun will remain in the center of the screen.
        The simulator will effectively track along with the sun's movement.

        \item The simulator will be mobile friendly.
	\end{enumerate}

\subsubsection{Eclipse Image Processor}
	\begin{enumerate}
		\item The image processor will be implemented as a class that will be easily
        inheritable / modifiable by developers.

        \item The image processor will feature two modes, ''window'' and ''batch''. In window
        mode, after an image is processed, windows will open showing the original image,
        processed image, and intermediate images. In batch mode, all images will be processed
        sequentially without opening any windows. Processed images and metadata will be exported
        as described in \textit{External Interfaces: Eclipse Image Processor}.

        \item The image processor will identify the circles of the sun/moon in the images it processes.

        \item The image processor will record the number of seconds (wall clock time) needed to
        complete various portions of each image's processing.

	\end{enumerate}

\subsubsection{Eclipse Image Processor Developer Pipeline}
	\begin{enumerate}
		\item The developer pipeline will be able to download images from Google Cloud Storage, if requested.

        \item The developer pipeline will build and invoke the image processor on the requested images
        using batch mode.

        \item The developer pipeline will assemble the results of running the image processor into
        and HTML file. This HTML file, along with the processed images it references,
        will be uploaded to Google Cloud Storage to a public URL.

        \item The HTML file created by the developer pipeline will summarize the data included
        in the metadata.txt file exported by the image processor.

	\end{enumerate}

% SECTION 3.3 %%%%%%%%%%%
\subsection{Performance Requirements}

\subsubsection{Eclipse Simulator}
	\begin{enumerate}
		\item All simulator resources will load in less than 700ms given a 1-10
		 Mbps internet connection.
	\end{enumerate}

\subsubsection{Eclipse Image Processor}
	\begin{enumerate}
		\item The application should take less than 5 seconds to process an
		 image when running on a Google Cloud Platform n1-standard-4 virtual
		 machine.
	\end{enumerate}

\subsubsection{Eclipse Image Processor Developer Pipeline}
	\begin{enumerate}
		\item The image processor developer pipeline will download/upload images from/to Google
        Cloud Storage in parallel.
	\end{enumerate}

%%%%%%%%%%%%%%%%%%%%%%%%%%%%%%%%%
%  SECTION 4: Supporting Information
%%%%%%%%%%%%%%%%%%%%%%%%%%%%%%%%%%
\newpage
\section{Supporting Information}

\subsection{Appendix}

\begin{enumerate}
	\item Figure 1. Project Gantt Chart
\end{enumerate}

\begin{center}
	\includegraphics[width=\textwidth]{gantt1.eps}
\end{center}

\end{document}

