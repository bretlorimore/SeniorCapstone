\documentclass[10pt, onecolumn, draftclsnofoot, letterpaper, compsoc]{IEEEtran}

\usepackage{cite}
\usepackage{hyperref}
%usepackage{enumitem}
\usepackage{graphicx}
\usepackage{longtable}

\graphicspath{ {images/} }

\renewcommand*\contentsname{Table of Contents} % Rename ToC

\newcommand{\myindent}{\hspace{\oldparindent}}

\usepackage{cite}

\title{Tech Review}
\date{\today} % Somehow this isn't working..
\author{Totality AweSun \\
		Bret~Lorimore, Jacob~Fenger, George~Harder \\
		\textit{November 14th, 2016 \\
		CS 461 - Fall 2016}}

\begin{document}

\maketitle

\begin{abstract}

This document compares various technologies for use in the North American Solar
Eclipse 2017 senior capstone project. Specifically, we look at three specific parts
of the project, the Eclipse Image Processor, the Image Processor Manager, and the
Eclipse Simulator. For each of those high level components, we review applicable
technologies for three of their sub-components. For each of the nine total sub-components,
we make a conclusion about which of the considered technologies is best suited to our
needs, and will thus be used in our implementation.

\end{abstract}

\newpage

\section{Introduction}

This technology review proceeds in three similarly structured sections. First,
George Harder discusses what technologies will be the most effective for the
implementation of the Eclipse Image Processor. This section is broken down into
three subsections that correspond to the pieces he is responsible for in this
project. The pieces are Image Classification and Manipulation, Runtime
Environment, and Circle Detection. In the next section Bret Lorimore discusses
his area of responsibility, the Image Processor Manager. The three pieces that
Bret examines are Downloading/ Uploading from Google Cloud Storage, Storing
Photo Metadata, and Application Parallelization. Lastly, Jacob Fenger details
the pieces and technologies that will comprise the Eclipse Simulator. The three
pieces he will be responsible for are User Interface; Gathering User
Latitude/Longitude Information; and Displaying Sun/Moon Based On Latitude,
Longitude, and Time Information. \\

An important note with regard to some of the technologies being discussed is
that Google is sponsoring this project and as such we have been asked to use
Google products when possible. This is especially relevant to the sections on
runtime environments and storing the photo metadata. Our sponsor's request that
we use Google products if possible is by no means limiting, and in fact makes
available to us extremely useful state of the art tools. \\


% George's Section
\section{Eclipse Image Processor}

The eclipse image processor is the piece of our project that whandles the spatial
and temporal alignment of the eclipse images that are uploaded to the Eclipse
Megamovie website. We have identified three pieces into which
the image processor can be broken down. These are: image classification and
manipulation, the runtime environment, and circle detection. In order for this
element of our project to operate effectively it is critical that these three
pieces utilize robust and functional technologies. This section of the
technology review details what options are available for implementing each of
these three pieces, analyzes these options, and arrives at a determination as to
which option is the best in the context of this project.\\

\subsection{Image Classification and Manipulation}

The image classification and manipulation portion of the eclipse image processor
encapsulates a few key requirements for this project. For our project to be
functionally complete it must determine temporal ordering of the images, align
them spatially, and crop them such that the sun is the same size in all of the
images. This process requires the use of a computer vision application
programming interface (API). Three possible options of APIs are: OpenCV,
SimpleCV, and VXL. \\

The relative strengths and weaknesses of these three options can be evaluated
using several key criteria that emerge when considering the requirements for
this element of the project. These criteria are: speed, support/documentation,
availability, ease of installation, and ease of use. Fortunately, all three of
these tools are available online and provide documentation on installation so
these categories do not require much discussion \cite{OCV, VXL, SCV}. With these
criteria met, and considering the performance requirements of our project, the
next most important criteria to consider is speed. \\

OpenCV and VXL both rank highly in the speed category because they are written
natively in C++\cite{OCV, VXL}. SimpleCV on the other hand is written in Python
and thus uses an interpreter \cite{SCV}, which causes speed penalties.
SimpleCV's use of Python is a significant drawback and it cannot be ignored.
Because of the large volume of images this tool is expected to process, we are
targeting a mean processing time of 1 second per image. This is not necessarily
impossible with an interpreted language, but it introduces major difficulties in
trying to meet this requirement. OpenCV and VXL are the clear winners when
considering speed. \\

While not as critical to meeting specific requirements as speed, it is important
to consider the costs associated with learning a new API. While all three APIs
provide online documentation, these docs are by no means equal. OpenCV and
Simple CV both provide easy to understand and what appear to be comprehensive
tutorials \cite{OCV, SCV}. While OpenCV may be more complicated than the aptly
named SimpleCV it has a massive number of contributors on question sites like
Stack Overflow \cite{stkovrflw}. Having an online community as a resource is an
invaluable asset when learning an API and this is one of the great strengths of
OpenCV. VXL, unlike the other two possibilities provides relatively poor and
hard to navigate documentation. The time required to learn how to use an API can
significantly slow down the development process of an application like our image
processor. Having high quality documentation and tutorials at our disposal, as
is the case with OpenCV and SimpleCV, is critical to surmounting the learning
curve and its importance should not be overlooked. \\

\begin{table}[h]
\centering
\caption{Comparison of Possible Technologies for Image Classification and Manipulation}
\begin{tabular}{|p{2.1cm}|p{2.1cm}|p{2.1cm}|p{2.1cm}|p{2.1cm}|p{2.1cm}|}
\cline{4-4}

\hline  & Available & Installation & Speed & Support and Documentation & Ease of
Use  \\ \hline

OpenCV  & Yes & Easy & Fast (native C++) &  Excellent. Tutorial, docs, huge
online community & Medium. May take some learning, offset by support/
documentation  \\ \hline

SimpleCV & Yes & Easy & Slow (native Python) & Good. Tutorial, docs, book
available for purchase & Easy. Meant to be simple by design.  \\ \hline

VXL & Yes & Easy & Fast (native C++) & Poor. Hard to read docs, not much support
online. & Difficult. Collection of libraries, poor docs, likely tough to learn.
\\ \hline

\end{tabular}
\label{table:george1}
\end{table}

\textit{Conclusion} [See summary of previous discussion in Table
\ref{table:george1}]: OpenCV and SimpleCV are highly comparable in terms of ease
of use and support/documentation. SimpleCV may be easier at first, but any
advantage it has is offset by the robust online community willing to support
each other and answer questions that OpenCV has. VXL falls flat when compared to
OpenCV and SimpleCV in those two categories. With this in mind, and when
considering that speed is of critical importance, OpenCV is the clear choice of
computer vision API for this project. \\

\subsection{Runtime Environment}

The high volume of images the Eclipse Megamovie project expects to collect
necessitates some form of scalability for our image processor. Without a runtime
environment that we can effectively scale to process hundreds of thousands or
perhaps millions of images in a timeframe that meets the requirements of the
larger system, the image processor's utility is near zero. We have identified
three potential options to meet the needs for scaling this image processor:
Docker containers, cloud based virtual machines (VMs), and Google Cloud
Functions (GCF).\\

In order to determine which runtime environment will best allow us achieve the
scalability necessary for the eclipse image processor several criteria will be
considered. In this case they are: availability, cost, integration with our
application, security, and overhead. Our group's partnership with Google for
this project produces a natural inclination toward using the Google Cloud
Platform for whichever of these options we end up using. Thus, all of these
products have equal availability and our sponsor has made it clear that any
monetary costs associated with using Google products are negligible. Because all
of the options rank equally in the availability and cost categories only the
other criteria will be discussed below. \\

As the Google Cloud Platform supports all three possible technologies we are
considering, integration with our eclipse image processor would appear to be
essentially equal across all three options. However, this is not the case. Using
cloud based VMs or Docker Containers allows us to encapsulate our application
in a manner that allows for rapid scaling in addition to easy retrieval and
transmission of images and data \cite{docker, gcp}. GCF on the other hand uses
an event based microservice approach to app deployment \cite{gcp}. Instead of
containing our application, GCF is a Javascript function that would call our
eclipse image processor whenever an event is triggered \cite{gcp}. In this case
the event would be a photo upload to the Eclipse Megamovie website. Rather than
encapsulating our application GCF would spawn a child process, the eclipse image
processor's executable, each time it was called. This results in a more complex
deployment strategy because we would need to determine how to retrieve, store,
and transmit data without dedicated virtual memory. \\

Security is of paramount importance when deploying a web based application that
will be handling a high volume of uploads, downloads, and user interactions.
User information being compromised as the result of a vulnerability in our
application would inevitably harm participation in any future projects similar
to this one. Fortunately, the Google Cloud Platform has a strong commitment to
security \cite{gcp}. Google Compute Engine VMs encrypt all of their data as it
is stored and transmitted \cite{gcp}. This is an enormous advantage in terms of
security because it allows us to focus on developing our application's
functionality rather than worrying about securing it. Docker containers are
deployed on top of VMs, so they have all of the same security advantages as a
Compute Engine VM. In addition to this, Docker containers are completely
isolated from the rest of the VM they are running on \cite{docker}, so if one
were compromised the issue could be handled in a manner that does not impact any
other containers. GCF has its own advantages in terms of security. Each function
runs in its own execution environment \cite{docker}. Like the Docker containers,
this means that all of the instances of GCF are separate from one another. GCF
is also running in what Google calls a "serverless" environment \cite{docker},
this means that GCF is not storing data or using hardware that could be
compromised. All of the options receive high marks in the security category. \\

In the context of evaluating these options, overhead refers to the amount of
infrastructure that would need to be maintained in order to support each option
as it scales to meet the demands of a high volume of uploads. For Google Compute
Engine VMs, the overhead is fairly high. Each VM requires its own guest
operating system, virtual memory, and virtual hardware to run an instance of our
application \cite{docker, gcf}. This is in contrast to Docker containers, which
can run multiple instances of an application on top of a single host
\cite{docker}. Each instance of the application only needs its dependencies to
run. This model requires less overhead in order to scale our application, which
in this case results in fewer places where things can go wrong. By far the most
lightweight approach, GCF operates using a serverless model \cite{gcp}. This
provides advantages to us as developers because it greatly simplifies
deployment. However, it presents new challenges in terms of how we retrieve and
transmit data. In addition to the challenges of data management, another
disadvantage of the serverless model is that it takes away our control over how
much compute power we can provide to the eclipse image processor. \\

\begin{table}[h]
\centering
\caption{Comparison of Possible Technologies for Runtime Environments}
\begin{tabular}{|p{2.1cm}|p{2.1cm}|p{2.1cm}|p{2.1cm}|p{2.1cm}|p{2.1cm}|}
\cline{4-4}

\hline  & Available & Cost & Integration & Security & Overhead \\ \hline

Virtual Machines & Yes & Neglible & Good, can encapsulate app and dependencies
easily &  Excellent & High, VM per instance has high cost  \\ \hline

Docker Containers  & Yes & Neglible & Good, provides encapsulation &  Excellent
& Medium, clusters deployed on top of VMs  \\ \hline

Google Cloud Functions  & Yes & Neglible & Okay, JS can call our executable but
introduces unnecessary complexity & Excellent & Low, lightweight serverless
Javascript function calls  \\ \hline

\end{tabular}
\label{table:george2}
\end{table}

\textit{Conclusion} [See summary of previous discussion in Table
\ref{table:george2}]: Taking all criteria into consideration, the best choice of runtime environment
for our application will be Docker containers. These retain the best elements of
the other options while shedding their defects. \\

\subsection{Circle Detection}

Detecting circles is an area of ongoing research in computer science and is of
particular interest to this project because it is necessary for the temporal
ordering and spatial alignment of the eclipse images. In order to determine the
temporal ordering of the images we plan to use the relative difference in
position of the centers of the sun and the moon. To spatially align the images
we will be rotating them so that the visible portion of the sun is at a fixed
angle relative to the center of the image. For either of these processes to be
successful we need a technique capable of locating circles and their centers. We
have identified three possible algorithms: Hough transforms, Blob detection and
the Ant System Algorithm. \\


In order to select an algorithm for circle detection, three primary criteria
must be considered. They are: computational complexity, accuracy, and
implementation difficulty. Computational complexity is of high importance
because a slow algorithm or one that uses a large amount of memory jeopardizes
our chances of meeting performance requirements and places strain on our runtime
environment. We need this algorithm to achieve high accuracy to ensure the
seamless integration of the eclipse image processor with the Eclipse Megamovie
project. If our algorithm cannot correctly identify circles, our processor could
not be considered functional. Lastly, an important consideration is the level of
difficulty required in integrating our algorithm of choice into the eclipse
image processor's codebase. \\


The Hough transform is a common algorithm for circle and line detection that is
known to have high computational complexity and memory usage \cite{hough,
antsystem}. This is generally disadvantageous to the performance of our eclipse
image processor. Blob detection has a lower computational complexity and memory
usage than the Hough transform because it does not need to store an array of all
of the possible circle centers like the Hough transform does \cite{hough,
blobarticle, blobref}. The least computationally complex algorithm is the Ant
System Algorithm described by Chattopadhyay et al. \cite{antsystem}. This
algorithm's complexity is determined solely by the number of pixels on the edge
of shapes in the image. This is significantly lower than the complexity of other
two algorithms which rely on the number of pixels in the whole image. \\


Accuracy is of critical importance to the success of the eclipse image
processor. As such, it is necessary that the circle detection algorithm
consistently detects the sun and moon in the eclipse images. Hough transforms
have been shown to be very successful in this regard \cite{imgKrista}. To this
group's knowledge, blob detection has not been shown to be effective at finding
the sun and moon in eclipse images. OpenCV does include a blob detector that can
filter blobs by their relative circularity and return their centers
\cite{blobarticle, blobref}, which meets the needs of this processor. However,
the accuracy of blob detection is likely to suffer in this application because
the pixels corresponding to the moon are likely to blend with the background of
the image since they are both black. This results in there being no distinct
blob to be detected in the crescent images. The Ant System Algorithm, while
shown to be accurate \cite{antsystem}, could fall victim to this same problem.
The images that the Ant System Algorithm were tested on had white outlined
shapes over a black background \cite{antsystem}, this is not how the real world
eclipse images will appear. \\


The final consideration in selecting a circle detection algorithm is the effort
it will take to use it in the eclipse image processor. Both the Hough transform
and blob detection score very highly in this category. OpenCV has
implementations of these methods, documentation on how to use them, and even has
code examples of their use \cite{houghocv, blobarticle}. This is extremely useful
and makes the implementation of the circle detection portion of the image
processor much simpler. The Ant System algorithm on the other hand would need to
be implemented by our team and would be based on pseudocode in the Chattopadhyay
et al. paper. This presents a major disadvantage because it jeopardizes both the
accuracy and performance of the algorithm. There is no guarantee that our
implementation of the algorithm will work as well as the one the author's of the
paper used. This also requires potentially several extra weeks of development
and testing. Having to implement our own version of the Ant System Algorithm
essentially eliminates it from contention as a possible circle detection
algorithm. \\

\begin{table}[h]
\centering
\caption{Comparison of Possible Technologies for Circle Detection}
\begin{tabular}{|p{3.15cm}|p{3.15cm}|p{3.15cm}|p{3.15cm}|}
\cline{4-4}

\hline  & Available & Cost & Integration \\ \hline

Hough Transform & High, uses a lot of memory and iterates over each pixel more
than once and performs a computation each step & Excellent, known to have good
accuracy and has been shown to work on eclipse images & Low, supported in OpenCV
\\ \hline

Blob Detection & Medium, iterates over each pixel once performing computations
at each step & Good, known to be accurate at finding circles and centers but
hasn't been known to work with eclipses & Low, supported in OpenCV \\ \hline


Ant System Algorithm & Low, has complexity O(edge pixels), could be compromised
by our implementation & Okay, not shown to work on real world images, could be
worsened by our implementation & Extreme, we would have to write our own
implementation based on pseudocode \\ \hline

\end{tabular}
\label{table:george3}
\end{table}

\textit{Conclusion} [See summary of previous discussion in Table
\ref{table:george3}]: While the Hough transform may struggle in terms of computational
complexity, it emerges as a clear favorite when accuracy and ease of
implementation are taken into account. It is the only algorithm known to
successfully detect the sun and moon in eclipse images and is supported in
OpenCV.

% Bret's Section
\section{Image Processor Manager}

The image processor manager will be a Python application responsible for
collecting/downloading user uploaded eclipse images to be processed by the image
processor application, invoking the image processor with these images as input,
and collecting the output of the image processor application and uploading it
to Google Cloud. In this section of the technology review we consider various
technologies for downloading/uploading photos from Google Cloud Storage where
they will be stored, several technologies to manage our photo metadata, and
different methods for incorporating parallelization into this application. We will
be evaluating these various technologies on the following criteria as applicable:
functionality, security, speed, ease of development, and ease of integration with
the rest of the project. \\

\subsection{Downloading/Uploading from Google Cloud Storage}

There are several methods we could use to download/upload images from Google Cloud
Storage. The first of these is the gsutil application. gsutil is part of the Google
Cloud SDK and makes it easy to list the contents of a storage bucket, download files
from a bucket, upload files to a bucket, and more from the command
line\cite{gsutil, cloudStorage}. For our purposes, authentication using
service accounts would be most suitable. Service accounts can be
thought of as user accounts for applications where the applications "log in"
using special authentication credentials instead of usernames/passwords. Configuring
service account authentication for gsutil requires enabling a particular service
account with the entire Google Cloud SDK using the gcloud application and a service
account credentials file. As gsutil is a command line utility, it would need to be
integrated with our app as a subprocess, using the Python {\tt subprocess} module. \\

Google Cloud Storage is also accessible via a JSON Rest API. Like gsutil, this API
allows users to easily list the contents of a storage bucket, download files from
a bucket, upload files to a bucket, and more via HTTP requests\cite{cloudStorageJSON}.
Since our application has its own Cloud Storage buckets, authentication for the
JSON API would be handled using service accounts\cite{cloudStorageJSON}. Integrating
the JSON API with our application would require use of a Python HTTP library and explicit creating
of all the required API request URLs\cite{cloudStorageJSON}. \\

The final method we consider for interacting with Google Cloud Storage is the
Google Cloud Client Library for Python. This client library is easily
installable using pip (the standard Python package manager) and once installed,
it allows programmatic access to Google Cloud Storage via a simple Python
API\cite{cloudStorageLib}. Authentication for this client library is, like with
the JSON API and gsutil, handled using service accounts\cite{cloudStorageLib}.
Setting up this authentication is simple and is done by setting a special
environment variable to hold the path to a credentials file, obtained from
Google\cite{cloudStorageLib}. As this client library provides a native Python
API, integrating it with out application would simply require importing the
necessary modules and embedding API calls directly in the source code via
Python method calls\cite{cloudStorageLib}. \\

\textit{Conclusion} [See summary of the above comparison in Table \ref{table:bret1}]:
The Google Cloud Client Library for Python is the best
option for our needs as it is easy to install, it is secure, it provides all the
required functionality, and is very simple to integrate with our application
source code. \\

\begin{table}[h]
\centering
\caption{Comparison of technologies for downloading/uploading from Google Cloud Storage}
\begin{tabular}{|p{1.8cm}|p{1.4cm}|p{3cm}|p{1.8cm}|p{1.8cm}|p{1.8cm}|p{1cm}|}
\cline{4-4}
\hline

% Row 1
 & Provides needed functionality? & Installation procedure & Authentication
 & Integration with our app & Ease of installation/ integration & Secure? \\ \hline

% Row 2
gsutil & yes & apt-get, copy service account credentials file, setup service
account as default auth method using gcloud & service accounts & invoke as
sub-process & medium, medium & yes \\ \hline

% Row 3
Cloud Storage JSON API & yes & install Python HTTP library, copy service
account credentials file & service accounts & make HTTP requests using
some Python HTTP library & easy, hard & yes \\ \hline

% Row 4
Google Cloud Client Library for Python & yes & pip, copy service account
credentials file, set environment variable & service accounts & native
integration into source code & medium, easy & yes \\ \hline

\end{tabular}
\label{table:bret1}
\end{table}

\subsection{Storing Photo Metadata}

For each photo we will store a metadata record containing various information
including the filename, whether or not the photo has been processed, and if it
has, output information from the image processor. The image processor manager
will use these records to know which photos to request from Cloud Storage
(ones that have not been processed), and these records will be updated after
processing to include the output of the image processor. \\

We will evaluate several potential technologies to facilitate this data
storage/manipulation. The first of these is Google Cloud Datastore. Datastore
is a simple NoSQL database solution \cite{cloudDatastore}. This means that it is
schemeless and typeless, so programmers must be careful to validate that the
data being inserted conforms to both the desired scheme and type.
Datastore is a fully managed solution so users do not need to worry about setting
up virtual machines or managing any of the other infrastructure associated with
running Datastore\cite{cloudDatastore}. Programmers can easily interact with Datastore
via the Google Cloud Client Library for Python that is mentioned in the previous
section\cite{cloudDatastoreDocs}. This library takes care of authenticating requests
using service accounts\cite{cloudDatastoreDocs}. The API additionally provides access
to Datastore transactions\cite{cloudDatastoreDocs}. This functionality is important for
our purposes, as we will need to request a list of image files and mark them as pending
processing before any other application can read a list containing any of these same files.
This behavior is necessary to ensure that multiple nodes do not pull down the same image
files to process. \\

The second solution we consider is Google Cloud Bigtable. Bigtable is a managed NoSQL
database solution and is similar to Datastore in many ways\cite{cloudBigtable}.
Bigtable is tailored to applications that need to store massive amounts of data
and query it very quickly\cite{cloudBigtable}. As such, it is a much more heavyweight
solution than Datastore. It offers users much more control than Datastore does.
Bigtable allows users to create their own clusters and specify the type of hardware on
which they will run\cite{cloudBigtableDocs}. Users can also create multiple tables
as necessary\cite{cloudBigtableDocs}. These are features that are not offered with
Datastore. As part of Google Cloud, Bigtable offers a Python API to interact with it
and handle authentication using service accounts\cite{cloudBigtableDocs}. Bigtable has
many of the same basic features as Datastore but offers users much more control
and access to much more powerful systems. Google does not recommend Bigtable if
you are working with less than 1TB of data\cite{cloudBigtable}. Additionally,
bigtable does not support transactions\cite{cloudBigtable}. \\

The last solution we investigate is running our own MySQL servers. This solution
would serve our needs as we would be able to store all our data and MySQL
does support transactions. That being said, it would be a great deal of work to
use our own MySQL servers. We would have to create/manage our own virtual machines,
manage database replication/backups, setup some sort of load balancer to send
request to the different MySQL nodes, and configure access controls which is
non-trivial to do securely. One upside of using MySQL is that it has a scheme
and is typed, so this relieves some of the pressure from programmers to
validate data based on scheme/type. In order to connect our Python code to MySQL,
we would need to incorporate an additional MySQL connector library into our
application\cite{mysql}. \\

\textit{Conclusion} [See summary of the above comparison in Table \ref{table:bret2}]:
Datastore will serve the needs of our project best as it is
very simple to setup and manage, is simple to integrate with our application,
and provides all the functionality we need. Despite not supporting transactions,
Bigtable could likely be made to work, however this would just be adding more
overhead in both setup and creating a transactions workaround. On top of this,
Bigtable is fundamentally not the right solution for our needs. It is
targeted at applications that store/query much more data than ours does. \\

\begin{table}[h]
\centering
\caption{Comparison of technologies for storing photo metadata}
\begin{tabular}{|p{1.8cm}|p{1.4cm}|p{3cm}|p{1.8cm}|p{1.8cm}|p{1.8cm}|p{1cm}|}
\cline{4-4}
\hline

% Row 1
 & Provides needed functionality? & Setup procedure & Authentication
 & Integration with our app & Ease of setup & Secure? \\ \hline

% Row 2
Google Cloud Datastore & yes & enable Datastore in Google Cloud project
& built in: service accounts & Google Cloud Client Library for Python
& easy & yes \\ \hline

% Row 3
Google Cloud Bigtable & not natively, no support for transactions &
create Bigtable cluster, required tables & built in: service accounts
& Google Cloud Client Library for Python & medium & yes \\ \hline

% Row 4
custom MySQL servers & yes & create virtual machines, install MySQL,
configure network access, configure access controls, setup backup/
replication credentials file, set environment variable & requires custom setup
& requires Python-MySQL connector & hard & yes, if done correctly \\ \hline

\end{tabular}
\label{table:bret2}
\end{table}

\subsection{Application Parallelization}

Downloading/uploading images from Google Cloud Storage and running the
image processor on a set of images are both relatively high latency operations.
For that reason, it is desirable for this application to do multiple operations
concurrently. Here we consider various solutions for doing this in Python. In
this comparison, we consider invoking the image processor binary using the
builtin Python {\tt subprocess} module. This module provides a nice Python wrapper
around standard {\tt fork}/{\tt exec}/{\tt waitpid}/etc. C functions. \\

The first solution we consider is standard multithreading using the builtin
Python {\tt threading} module. This can be used to both invoke multiple image
processor processes concurrently and to make concurrent requests to Google
Cloud Storage and Datastore. A common issue with use of the Python {\tt threading}
module is the Python Global Interpreter Lock (GIL)\cite{gilArticle}. The GIL
is a global lock in the CPython interpreter that allows only one thread to
execute code in the interpreter at a time\cite{gilArticle}. The GIL would
not cause us many issues when invoking the image processor binary, as
the Python subprocess implementation releases the GIL before calling
waitpid on a particular sub-process. This means that the
setting up/calling of a particular invocation of the image processor binary
would be done serially, but the actual processing could be done currently as
it is done outside of the Python interpreter and the GIL is released\cite{gilArticle}.
The GIL is also released when making socket calls, so while setup and
pre/post-processing of Cloud Storage uploads/downloads would be
serialized, the socket calls themselves could be working in parallel\cite{gilArticle}.
Implementation using the {\tt threading} module requires explicit
creation/starting/joining of {\tt Thread} objects. \\

An alternative solution is to use the builtin Python {\tt multiprocessing.Pool}
class. This method sidesteps the GIL entirely by using processes for
parallelism instead of threads\cite{multiproc}. This method is also desirable
as the API is incredibly simple - a pool of n processes can be created in a
single line and then these processes can be passed a target function and
data in a single other line of code\cite{multiproc}. The only requirement
here is that all parameters passed to the process pool must be picklable\cite{multiproc}.
This would not be an issue for us as our data will take the form of standard
Python datatypes. This approach does require creating entire new processes,
however on Linux this can be done very very quickly. In practice, regardless
of the application, sidestepping the GIL and implementing concurrency using
processes instead of threads is almost always the faster solution in Python. \\

The last parallelization solution we investigate is applicable only to calls to
Google Cloud Datastore. This is using batch requests instead of single requests.
The Datastore API has built in batch request functionality which allows multiple
queries to made in a single API call, meaning only a single HTTP request is made
for all grouped the queries\cite{batches}. When possible, this is desirable over
multiple concurrent requests as there is reduced overhead with establishing
TCP connections. This application will never be sending/receiving large
pieces of data to Datastore, so the savings associated with establishing
far fewer TCP connections is highly desirable. \\

\textit{Conclusion} [See summary of the above comparison in Table \ref{table:bret3}]:
The {\tt multiprocessing.Pool} class is the best solution for us
for invoking the image processor and uploading/downloading from Google Cloud
Storage, as it is very simple to integrate and will almost certainly offer
higher performance than using the {\tt threading} module. Using batch requests for
Datastore calls is preferable to creating concurrent requests using the
{\tt multiprocessing.Pool} class as with batch requests we only have the overhead
of creating a single TCP connection for all the datastore queries. \\

\begin{table}[h]
\centering
\caption{Comparison of technologies for application parallelism}
\begin{tabular}{|p{4.2cm}|p{4.2cm}|p{4.2cm}|}
\cline{3-3}
\hline

% Row 1
 & Applicable to & Integration Overhead \\ \hline

% Row 2
{\tt threading} & image processor invocation, GCS/Datastore requests
& medium: requires explicit thread creation/starting/joining,
requires consciousness of shared data \\ \hline

% Row 3
{\tt multiprocessing.Pool} & image processor invocation, GCS/Datastore requests
& medium: simple to invoke, requires pickleable parameters, cannot rely
on shared data unless we use IPC \\ \hline

% Row 4
Batch requests & Datastore requests & low, requires coalescing multiple
requests into single request  \\ \hline

\end{tabular}
\label{table:bret3}
\end{table}

% Jake's Section
\section{Eclipse Simulator}
The eclipse simulator will be a standalone JavaScript module enabling users to
"preview" the eclipse. It will be designed in a stylized, 2D manner.
The simulator will incorporate a time slider to allow users to simulate
the eclipse in a time window spanning from 12 hours before the eclipse to 12
hours after it. As users drag the time slider, the eclipse will animate in the
simulator window. The view of the eclipse which users are presented will be
specific to a location that the user enters. Additionally, the time will be
displayed in the simulator based on what location the user enters and the
positioning of the time slider. \\

\subsection{User Interface}

The user interface for this web based simulator will use 2D animated depictions
of the Sun and the Moon as they appear at a user specified time and location.
Additionally, the user interface will contain background imagery such as a
city or hillside landscape. There will also be a time slider, a location
input, and a time display for user's to interact with or view. \\

For the first possible method, the most basic approach would be to utilize
HTML, CSS, and JavaScript for output and input. Altering the locations of the
sun and moon would be done by repositioning images of the sun and moon based on
location data. While this approach can be simple, repositioning images and
making the interface look appealing to the user can be difficult or
inefficient. \\

Another method would be to utilize HTML5's Canvas graphics API. Canvas is
capable of rendering graphics directly to the screen via JavaScript. This is
otherwise known as "immediate mode" graphics, which means that the rendered
graphics are not saved \cite{SVGvsC}. Additionally, Canvas draws individual
pixels to the screen. For this simulator, the only image objects that we need
display differently depending on user input are the Sun and Moon images.
Since these are relatively small compared to rest of the simulator,
re-rendering the Sun and Moon most likely will not result in poor performance. \\

The last technology for displaying the simulator would be to use another HTML5
component called Scalar Vector Graphics (SVG). This is a shape based method
for describing 2D graphics via XML.  SVG utilizes objects to describe images.
The browser is capable of re-rendering shapes automatically if attributes to
an object are changed \cite{SVG}. Adding sliders or round buttons to the
simulator can also be a straightforward approach with vector graphics. \\

For the user interface, speed is the most important criteria. All graphics and
other simulator resources will need to load in less than 500ms given a 1-10
Mbps internet connection. Additionally the animation used in the simulator
should be around 60 frames per second. \\

\textit{Conclusion} [See summary of previous discussion in Table \ref{table:Jake1}]:
SVG seems to be a better technology to use when creating
the user interface because of the interactivity that it has to offer for users.
This is due to the fact SVG can process events for each sub-element separately
while Canvas must process events for the entire canvas. Additionally, since
our simulator requires a few number of objects, we will see better performance
benefits than compared to the Canvas or scratch approach. \\

\begin{table}[h]
\centering
\caption{Comparison of Possible Technologies for User Interface}
\begin{tabular}{|p{3.15cm}|p{2cm}|p{2cm}|p{3.15cm}|p{3.15cm}|}
\cline{5-4}

\hline  & Model & Method & Difficulty of Implementation & Performance\\ \hline

Implement from Scratch & Images & HTML, JavaScript, CSS & May take time to
start from scratch & Dependent upon implementation, requires a lot of
bandwidth to send large files \\ \hline

Canvas & Pixel Based & JavaScript & Built into HTML5 & Smaller surface,
larger number of items \\ \hline

SVG & Vector Based & JavaScript, CSS & Built into HTML5 & Larger surface,
smaller number of items \\ \hline

\end{tabular}
\label{table:Jake1}
\end{table}

\subsection{Gathering User Latitude/Longitude Information}

For the eclipse simulator to be as accurate as possible, we must gather user
submitted location data for our computations. There are several options to
consider when implementing this. The first of which is having the user only be
allowed to enter a specific latitude and longitude when entering their
location. While this would be easy to do, it provides a negative user
interaction since people generally do not know the specific latitude and
longitude of places they will be located for the solar eclipse. \\

The second option would be to utilize the Google Maps API to display a map
\cite{jsAPI}.
The user would then place a pin at the location that they want the simulator
to be based off of. This would require a combination of HTML, CSS, and
JavaScript. Displaying a map for the user to interact act with is a much
better form of input than the first method. Users don't need to know the
latitude and longitude of certain locations, they just need to recognize
them via a map. Implementing this would be done by adding a map to the
simulator, and then adding a click event function to save the latitude and
longitude for when a user clicks a location on the map. \\

Lastly, we could utilize the Geocoding API which is built into the Google Maps
API for converting addresses into geographic coordinates \cite{jsGEO}.
The user could enter a wide range of input including specific addresses,
roads, zip codes, or other options. This data would then be geocoded and will
result in the latitude and longitude. \\

\textit{Conclusion} [See summary of previous discussion in Table \ref{table:Jake2}]:
Overall, geocoding user entered addresses will provide the
best functionality. The Google Maps Geocoding API can easily convert these
addresses to latitude and longitude coordinates for computing the Sun and
Moon's location. While dropping a pin on a map is simple, it requires more
another large space for user's to interact with and loading an entire
map may degrade performance. \\

\begin{table}[h]
\centering
\caption{Comparison of Possible Technologies for Lat/Long Information}
\begin{tabular}{|p{3.15cm}|p{2cm}|p{2cm}|p{2cm}|p{3.15cm}|}
\cline{5-4}

\hline  & Provides Needed Functionality? & User Friendly? & Ease of Setup
 & API \\ \hline

User Inputs Lat/Long & Yes & No & Very Easy & None \\ \hline

Drop Pin on a Map & Yes & Yes & Medium & Google Maps API \\ \hline

Geocoding & Yes & Yes & Medium & Google Maps Geocoding API \\ \hline

\end{tabular}
\label{table:Jake2}
\end{table}

\subsection{Displaying Sun/Moon Based on Latitude, Longitude, and Time
 Information}

One of the trickier aspects of the eclipse simulator is altering the positions
of the Sun and the Moon based on user entered location and time. The best metric
for observing locations of celestial objects is altitude and azimuth. The
altitude of an object is the distance it appears above the horizon, while the
azimuth is angular distance along the horizon to the object \cite{ALTAZI}.
It is also possible for the Sun and the Moon to different sizes. We may need
to alter the size of the Moon based on location and time while using the
simulator. The first method would be to write a JavaScript implementation
from scratch for computing the positions of the Sun and Moon based on
location and time. This could take significant time to ensure that all
equations are correct as well as confirming accurate results. \\

Another method would utilize a JavaScript library called SunCalc. This can be
used to compute sun position, sunlight phases, moon position, and lunar phases
based off location and time \cite{SunCalc}. This backend library is based off
of an article about positions of the Sun as seen from a planet \cite{AST}.
The documentation is also well written, which makes it very easy to use. \\

For the last method, another JavaScript library exists called Ephemeris
\cite{Ephem}. This is very similar to the previous method except that this
can be used for ephemeris calculations for a wider range of celestial objects.
Additionally, the documentation for this library is very lackluster compared
to the SunCalc library. \\

\textit{Conclusion} [See summary of previous discussion in Table \ref{table:Jake3}]: 
In conclusion, the SunCalc library seems to be the best
choice. There are several examples for how to use this library located on
the readme of the page \cite{SunCalc}. Additionally, utilizing this backend to
compute the Sun or the Moon's altitude and azimuth at a certain location
and time requires very few function calls. \\

\begin{table}[h]
\centering
\caption{Comparison of Possible Technologies for Sun/Moon Display}
\begin{tabular}{|p{3.15cm}|p{2cm}|p{2cm}|p{3.15cm}|}
\cline{4-4}

\hline  & Ease of Implementation & External Libaries? & Integration Difficulty
 \\ \hline

Backend from Scratch & Very Hard & None & Easy - this would be a tailored
solution \\ \hline

SunCalc Library & Easy & SunCalc & Easy \\ \hline

Ephemeris Library & Easy & Ephemeris & Easy \\ \hline

\end{tabular}
\label{table:Jake3}
\end{table}

%%%%%%%%%%%%%%%%%%%%%%%%%%%% END OF MAIN CONTENT %%%%%%%%%%%%%%%%%%%%%%%%%%%%%%
\newpage
\bibliographystyle{IEEEtran}
\bibliography{tech}

\end{document}

