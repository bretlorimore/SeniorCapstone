\documentclass[10pt, onecolumn, draftclsnofoot, letterpaper, compsoc]{IEEEtran}

\usepackage{cite}
\usepackage{hyperref}
%usepackage{enumitem}
\usepackage{graphicx}

\graphicspath{ {images/} }

%%%%%%%%%%%%%%%%%%%%%%%%%%%%%%%%%%%%%%%%%%%%%%%%%%%%%
% Macro for the signatures at the end                %
\newcommand*{\SignatureAndDate}[1]{%
    \par\noindent\makebox[2.5in]{\hrulefill} \hfill\makebox[2.0in]{\hrulefill}%
    \par\noindent\makebox[2.5in][l]{#1}      \hfill\makebox[2.0in][l]{Date}%
}%

\renewcommand*\contentsname{Table of Contents} % Rename ToC

\newcommand{\myindent}{\hspace{\oldparindent}}

\usepackage{cite}

% Temp title and author
\title{Requirements}
\date{\today} % Somehow this isn't working..
\author{Totality AweSun \\
		Bret~Lorimore, Jacob~Fenger, George~Harder \\
		\textit{November 4th, 2016 \\
		CS 461 - Fall 2016}}

\begin{document}

%\setlist[itemize]{topsep=1pt} % EDIT LISTS

\maketitle

% George's Section
\section{Eclipse Image Processor}

The eclipse image processor is the piece of our project that handles the spatial
and temporal alignment of the eclipse images that are uploaded to the Eclipse
Megamovie website. We have identified three pieces into which
the image processor can be broken down. These are: image classification and
manipulation, the runtime environment, and circle detection. In order for this
element of our project to operate effectively it is critical that these three
pieces utilize robust and functional technologies. This section of the
technology review details what options are available for implementing each of
these three pieces, analyzes these options, and arrives at a determination as to
which option is the best in the context of this project.

\subsection{Image Classification and Manipulation}

The image classification and manipulation portion of the eclipse image processor
encapsulates a few key requirements for this project. For our project to be
functionally complete it must determine temporal ordering of the images, align
them spatially, and crop them such that the sun is the same size in all of the
images. This process requires the use of a computer vision application
programming interface (API). Three possible options of API's are: OpenCV,
SimpleCV, and VXL. \\

The relative strengths and weaknesses of these three options can be evaluated
using several key criteria that emerge when considering the requirements for
this element of the project. These criteria are: speed, support/documentation,
availability, ease of installation, and ease of use. Fortunately, all three of
these tools are available online and provide documentation on installation so
these categories do not require much discussion \cite{OCV, VXL, SCV}. With these
criteria met, and considering the performance requirements of our project, the
next most important criteria to consider is speed. \\

OpenCV and VXL both rank highly in the speed category because they are written
natively in C++\cite{OCV, VXL}. SimpleCV on the other hand is written in Python
and thus uses an interpreter \cite{SCV}, which causes speed penalties.
SimpleCV’s use of Python is a significant drawback and it cannot be ignored.
Because of the large volume of images this tool is expected to process, we are
targeting a mean processing time of 1 second per image. This is not necessarily
impossible with an interpreted language, but it introduces major difficulties in
trying to meet this requirement. OpenCV and VXL are the clear winners when
considering speed. \\

While not as critical to meeting specific requirements as speed, it is important
to consider the costs associated with learning a new API. While all three API's
provide online documentation, these docs are by no means equal. OpenCV and
Simple CV both provide easy to understand and what appear to be comprehensive
tutorials \cite{OCV, SCV}. While OpenCV may be more complicated than the aptly
named SimpleCV it has a massive number of contributors on question sites like
Stack Overflow \cite{stkovrflw}. Having an online community as a resource is an
invaluable asset when learning an API and this is one of the great strengths of
OpenCV. VXL, unlike the other two possibilities provides relatively poor and
hard to navigate documentation. The time required to learn how to use an API can
significantly slow down the development process of an application like our image
processor. Having high quality documentation and tutorials at our disposal, as
is the case with OpenCV and SimpleCV, is critical to surmounting the learning
curve and its importance should not be overlooked. \\

\begin{table}[h]
\centering
\caption{Comparison of Possible Technologies for Image Classification and Manipulation}
\begin{tabular}{|p{1.8cm}|p{1.8cm}|p{1.8cm}|p{1.8cm}|p{1.8cm}|p{1.8cm}|}
\cline{4-4}

\hline  & Available & Installation & Speed & Support and Documentation & Ease of
Use  \\ \hline

OpenCV  & Yes & Easy & Fast (native C++) &  Excellent. Tutorial, docs, huge
online community & Medium. May take some learning, offset by support/
documentation  \\ \hline

SimpleCV & Yes & Easy & Slow (native Python) & Good. Tutorial, docs, book
available for purchase & Easy. Meant to be simple by design.  \\ \hline

VXL & Yes & Easy & Fast (native C++) & Poor. Hard to read docs, not much support
online. & Difficult. Collection of libraries, poor docs, likely tough to learn.
\\ \hline

\end{tabular}
\label{table:george1}
\end{table}

\textit{Conclusion} [See summary of previous discussion in Table
\ref{table:george1}]: OpenCV and SimpleCV are highly comparable in terms of ease
of use and support/documentation. SimpleCV may be easier at first, but any
advantage it has is offset by the robust online community willing to support
each other and answer questions that OpenCV has. VXL falls flat when compared to
OpenCV and SimpleCV in those two categories. With this in mind, and when
considering that speed is of critical importance, OpenCV is the clear choice of
computer vision API for this project.

\subsection{Runtime Environment}

The high volume of images the Eclipse Megamovie project expects to collect
necessitates some form of scalability for our image processor. Without a runtime
environment that we can effectively scale to process hundreds of thousands or
perhaps millions of images in a timeframe that meets the requirements of the
larger system, the image processor's utility is near zero. We have identified
three potential options to meet the needs for scaling this image processor:
Docker containers, cloud based virtual machines (VM's), and Google Cloud
Functions (GCF).\\

In order to determine which runtime environment will best allow us achieve the
scalability necessary for the eclipse image processor several criteria will be
considered. In this case they are: availability, cost, integration with our
application, security, and overhead. Our group's partnership with Google for
this project produces a natural inclination toward using the Google Cloud
Platform for whichever of these options we end up using. Thus, all of these
products have equal availability and our sponsor has made it clear that any
monetary costs associated with using Google products are negligible. Because all
of the options rank equally in the availability and cost categories only the
other criteria will be discussed below. \\

As the Google Cloud Platform supports all three possible technologies we are
considering, integration with our eclipse image processor would appear to be
essentially equal across all three options. However, this is not the case. Using
cloud based VM's or Docker Containers allows us to encapsulate our application
in a manner that allows for rapid scaling in addition to easy retrieval and
transmission of images and data \cite{docker, gcp}. GCF on the other hand uses
an event based microservice approach to app deployment \cite{gcp}. Instead of
containing our application, GCF is a Javascript function that would call our
eclipse image processor whenever an event is triggered \cite{gcp}. In this case
the event would be a photo upload to the Eclipse Megamovie website. Rather than
encapsulating our application GCF would spawn a child process, the eclipse image
processor's executable, each time it was called. This results in a more complex
deployment strategy because we would need to determine how to retrieve, store,
and transmit data without dedicated virtual memory. \\

Security is of paramount importance when deploying a web based application that
will be handling a high volume of uploads, downloads, and user interactions.
User information being compromised as the result of a vulnerability in our
application would inevitably harm participation in any future projects similar
to this one. Fortunately, the Google Cloud Platform has a strong commitment to
security \cite{gcp}. Google Compute Engine VM's encrypt all of their data as it
is stored and transmitted \cite{gcp}. This is an enormous advantage in terms of
security because it allows us to focus on developing our application's
functionality rather than worrying about securing it. Docker containers are
deployed on top of VM's, so they have all of the same security advantages as a
Compute Engine VM. In addition to this, Docker containers are completely
isolated from the rest of the VM they are running on \cite{docker}, so if one
were compromised the issue could be handled in a manner that does not impact any
other containers. GCF has its own advantages in terms of security. Each function
runs in its own execution environment \cite{docker}. Like the Docker containers,
this means that all of the instances of GCF are separate from one another. GCF
is also running in what Google calls a ``serverless” environment \cite{docker},
this means that GCF is not storing data or using hardware that could be
compromised. All of the options receive high marks in the security category. \\

In the context of evaluating these options, overhead refers to the amount of
infrastructure that would need to be maintained in order to support each option
as it scales to meet the demands of a high volume of uploads. For Google Compute
Engine VM’s, the overhead is fairly high. Each VM requires its own guest
operating system, virtual memory, and virtual hardware to run an instance of our
application \cite{docker, gcf}. This is in contrast to Docker containers, which
can run multiple instances of an application on top of a single host
\cite{docker}. Each instance of the application only needs its dependencies to
run. This model requires less overhead in order to scale our application, which
in this case results in fewer places where things can go wrong. By far the most
lightweight approach, GCF operates using a serverless model \cite{gcp}. This
provides advantages to us as developers because it greatly simplifies
deployment. However, it presents new challenges in terms of how we retrieve and
transmit data. In addition to the challenges of data management, another
disadvantage of the serverless model is that it takes away our control over how
much compute power we can provide to the eclipse image processor. \\

%Table Here

Taking all criteria into consideration, the best choice of runtime environment
for our application will be Docker containers. These retain the best elements of
the other options while shedding their defects.

\subsection{Circle Detection}

Detecting circles is an area of ongoing research in computer science and is of
particular interest to this project because it is necessary for the temporal
ordering and spatial alignment of the eclipse images. In order to determine the
temporal ordering of the images we plan to use the relative difference in
position of the centers of the sun and the moon. To spatially align the images
we will be rotating them so that the visible portion of the sun is at a fixed
angle relative to the center of the image. For either of these processes to be
successful we need a technique capable of locating circles and their centers. We
have identified three possible algorithms: Hough transforms, Blob detection and
the Ant System Algorithm. \\


In order to select an algorithm for circle detection, three primary criteria
must be considered. They are: computational complexity, accuracy, and
implementation difficulty. Computational complexity is of high importance
because a slow algorithm or one that uses a large amount of memory jeopardizes
our chances of meeting performance requirements and places strain on our runtime
environment. We need this algorithm to achieve high accuracy to ensure the
seamless integration of the eclipse image processor with the Eclipse Megamovie
project. If our algorithm cannot correctly identify circles, our processor could
not be considered functional. Lastly, an important consideration is the level of
difficulty required in integrating our algorithm of choice into the eclipse
image processor’s codebase. \\


The Hough transform is a common algorithm for circle and line detection that is
known to have high computational complexity and memory usage \cite{hough,
antsystem}. This is generally disadvantageous to the performance of our eclipse
image processor. Blob detection has a lower computational complexity and memory
usage than the Hough transform because it does not need to store an array of all
of the possible circle centers like the Hough transform does \cite{hough,
blobarticle, blobref}. The least computationally complex algorithm is the Ant
System Algorithm described by Chattopadhyay et al. \cite{antsystem}. This
algorithm’s complexity is determined solely by the number of pixels on the edge
of shapes in the image. This is significantly lower than the complexity of other
two algorithms which rely on the number of pixels in the whole image. \\


Accuracy is of critical importance to the success of the eclipse image
processor. As such, it is necessary that the circle detection algorithm
consistently detects the sun and moon in the eclipse images. Hough transforms
have been shown to be very successful in this regard \cite{imgKrista}. To this
group’s knowledge, blob detection has not been shown to be effective at finding
the sun and moon in eclipse images. OpenCV does include a blob detector that can
filter blobs by their relative circularity and return their centers
\cite{blobarticle, blobref}, which meets the needs of this processor. However,
the accuracy of blob detection is likely to suffer in this application because
the pixels corresponding to the moon are likely to blend with the background of
the image since they are both black. This results in there being no distinct
blob to be detected in the crescent images. The Ant System Algorithm, while
shown to be accurate \cite{antsystem}, could fall victim to this same problem.
The images that the Ant System Algorithm were tested on had white outlined
shapes over a black background \cite{antsystem}, this is not how the real world
eclipse images will appear. \\


The final consideration in selecting a circle detection algorithm is the effort
it will take to use it in the eclipse image processor. Both the Hough transform
and blob detection score very highly in this category. OpenCV has
implementations of these methods, documentation on how to use them, and even has
code examples of their use \cite{houghocv, blobarticle}. This is extremely useful
and makes the implementation of the circle detection portion of the image
processor much simpler. The Ant System algorithm on the other hand would need to
be implemented by our team and would be based on pseudocode in the Chattopadhyay
et al. paper. This presents a major disadvantage because it jeopardizes both the
accuracy and performance of the algorithm. There is no guarantee that our
implementation of the algorithm will work as well as the one the author’s of the
paper used. This also requires potentially several extra weeks of development
and testing. Having to implement our own version of the Ant System Algorithm
essentially eliminates it from contention as a possible circle detection
algorithm.

%Table Here

In conclusion, while the Hough transform may struggle in terms of computational
complexity, it emerges as a clear favorite when accuracy and ease of
implementation are taken into account. It is the only algorithm known to
successfully detect the sun and moon in eclipse images and is supported in
OpenCV.

% Bret's Section
\section{Image Processor Manager}

% Jake's Section
\section{Eclipse Simulator}


\bibliographystyle{IEEEtran}
\bibliography{tech}

\end{document}
