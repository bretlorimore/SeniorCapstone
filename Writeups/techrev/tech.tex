\documentclass[10pt, onecolumn, draftclsnofoot, letterpaper, compsoc]{IEEEtran}

\usepackage{cite}
\usepackage{hyperref}
%usepackage{enumitem}
\usepackage{graphicx}
\usepackage{longtable}

\graphicspath{ {images/} }

\renewcommand*\contentsname{Table of Contents} % Rename ToC

\newcommand{\myindent}{\hspace{\oldparindent}}

\usepackage{cite}

\title{Tech Review}
\date{\today} % Somehow this isn't working..
\author{Totality AweSun \\
		Bret~Lorimore, Jacob~Fenger, George~Harder \\
		\textit{November 14th, 2016 \\
		CS 461 - Fall 2016}}

\begin{document}

%\setlist[itemize]{topsep=1pt} % EDIT LISTS

\maketitle

% George's Section
\section{Eclipse Image Processor}

The eclipse image processor is the piece of our project that handles the spatial
and temporal alignment of the eclipse images that are uploaded to the Eclipse
Megamovie website. We have identified three pieces\cite{imgKrista} into which
the image processor can be broken down. These are: image classification and
manipulation, the runtime environment, and circle detection. In order for this
element of our project to operate effectively it is critical that these three
pieces utilize robust and functional technologies. This section of the
technology review details what options are available for implementing each of
these three pieces, analyzes these options, and arrives at a determination as to
which option is the best in the context of this project.

% Bret's Section
\section{Image Processor Manager}

The image processor manager will be a Python application responsible for  
collecting/downloading user uploaded eclipse images to be processed by the image
processor application, invoking the image processor with these images as input,
and collecting the output of the image processor application and uploading it
to Google Cloud. In this section of the technology review we consider various
technologies for downloading/uploading photos from Google Cloud Storage where
they will be stored, several technologies to manage our photo metadata, and
different methods for incorporating parallelization into this application. We will
be evaluating these various technologies on the following criteria as applicable:
functionality, security, speed, ease of development, and ease of integration with
the rest of the project.

\subsection{Downloading/Uploading from Google Cloud Storage}

There are several methods we could use to download/upload images from Google Cloud
Storage. The first of these is the gsutil application. gsutil is part of the Google
Cloud SDK and makes it easy to list the contents of a storage bucket, download files
from a bucket, upload files to a bucket, and more from the command 
line\cite{gsutil, cloudStorage}. For our purposes, authentication using
service accounts would be most suitable. Service accounts can be
thought of as user accounts for applications where the applications "log in"
using special authentication credentials instead of usernames/passwords. Configuring
service account authentication for gsutil requires enabling a particular service
account with the entire Google Cloud SDK using the gcloud application and a service
account credentials file. As gsutil is a command line utility, it would need to be
integrated with our app as a subprocess, using the Python subprocess module.

Google Cloud Storage is also accessible via a JSON Rest API. Like gsutil, this API 
allows users to easily list the contents of a storage bucket, download files from 
a bucket, upload files to a bucket, and more via HTTP requests\cite{cloudStorageJSON}.
Since our application has its own Cloud Storage buckets, authentication for the
JSON API would be handled using service accounts\cite{cloudStorageJSON}. Integrating 
the JSON API with our application would require use of a Python HTTP library and explicit creating
of all the required API request URLs\cite{cloudStorageJSON}.

The final method we consider for interacting with Google Cloud Storage is the
Google Cloud Client Library for Python. This client library is easily
installable using pip (the standard Python package manager) and once installed,
it allows programmatic access to Google Cloud Storage via a simple Python
API\cite{cloudStorageLib}. Authentication for this client library is, like with 
the JSON API and gsutil, handled using service accounts\cite{cloudStorageLib}. 
Setting up this authentication is simple and is done by setting a special 
environment variable to hold the path to a credentials file, obtained from 
Google\cite{cloudStorageLib}. As this client library provides a native Python 
API, integrating it with out application would simply require importing the 
necessary modules and embedding API calls directly in the source code via 
Python method calls\cite{cloudStorageLib}.

\textit{Conclusion} [See summary of the above comparison in Table \ref{table:bret1}]:
The Google Cloud Client Library for Python is the best
option for our needs as it is easy to install, it is secure, it provides all the
required functionality, and is very simple to integrate with our application 
source code.

\begin{table}[h]
\centering
\caption{Comparison of technologies for downloading/uploading from Google Cloud Storage}
\begin{tabular}{|p{1.8cm}|p{1.4cm}|p{3cm}|p{1.8cm}|p{1.8cm}|p{1.8cm}|p{1cm}|}
\cline{4-4}
\hline

% Row 1
 & Provides needed functionality? & Installation procedure & Authentication 
 & Integration with our app & Ease of installation/ integration & Secure? \\ \hline

% Row 2
gsutil & yes & apt-get, copy service account credentials file, setup service 
account as default auth method using gcloud & service accounts & invoke as 
sub-process & medium, medium & yes \\ \hline

% Row 3
Cloud Storage JSON API & yes & install Python HTTP library, copy service 
account credentials file & service accounts & make HTTP requests using 
some Python HTTP library & easy, hard & yes \\ \hline

% Row 4
Google Cloud Client Library for Python & yes & pip, copy service account 
credentials file, set environment variable & service accounts & native 
integration into source code & medium, easy & yes \\ \hline

\end{tabular}
\label{table:bret1}
\end{table}

\subsection{Storing Photo Metadata}

For each photo we will store a metadata record containing various information 
including the filename, whether or not the photo has been processed, and if it 
has, output information from the image processor. The image processor manager
will use these records to know which photos to request from Cloud Storage 
(ones that have not been processed), and these records will be updated after 
processing to include the output of the image processor.

We will evaluate several potential technologies to facilitate this data 
storage/manipulation. The first of these is Google Cloud Datastore. Datastore 
is a simple NoSQL database solution \cite{cloudDatastore}. This means that it is 
schemeless and typeless, so programmers must be careful to validate that the 
data being inserted conforms to both the desired scheme and type. 
Datastore is a fully managed solution so users do not need to worry about setting 
up virtual machines or managing any of the other infrastructure associated with 
running Datastore\cite{cloudDatastore}. Programmers can easily interact with Datastore 
via the Google Cloud Client Library for Python that is mentioned in the previous 
section\cite{cloudDatastoreDocs}. This library takes care of authenticating requests 
using service accounts\cite{cloudDatastoreDocs}. The API additionally provides access 
to Datastore transactions\cite{cloudDatastoreDocs}. This functionality is important for 
our purposes, as we will need to request a list of image files and mark them as pending 
processing before any other application can read a list containing any of these same files. 
This behavior is necessary to ensure that multiple nodes do not pull down the same image 
files to process.

The second solution we consider is Google Cloud Bigtable. Bigtable is a managed NoSQL 
database solution and is similar to Datastore in many ways\cite{cloudBigtable}. 
Bigtable is tailored to applications that need to store massive amounts of data 
and query it very quickly\cite{cloudBigtable}. As such, it is a much more heavyweight 
solution than Datastore. It offers users much more control than Datastore does. 
Bigtable allows users to create their own clusters and specify the type of hardware on 
which they will run\cite{cloudBigtableDocs}. Users can also create multiple tables 
as necessary\cite{cloudBigtableDocs}. These are features that are not offered with 
Datastore. As part of Google Cloud, Bigtable offers a Python API to interact with it 
and handle authentication using service accounts\cite{cloudBigtableDocs}. Bigtable has 
many of the same basic features as Datastore but offers users much more control 
and access to much more powerful systems. Google does not recommend Bigtable if 
you are working with less than 1TB of data\cite{cloudBigtable}. Additionally, 
bigtable does not support transactions\cite{cloudBigtable}.

The last solution we investigate is running our own MySQL servers. This solution 
would serve our needs as we would be able to store all our data and MySQL 
does support transactions. That being said, it would be a great deal of work to 
use our own MySQL servers. We would have to create/manage our own virtual machines, 
manage database replication/backups, setup some sort of load balancer to send 
request to the different MySQL nodes, and configure access controls which is 
non-trivial to do securely. One upside of using MySQL is that it has a scheme 
and is typed, so this relieves some of the pressure from programmers to 
validate data based on scheme/type. In order to connect our Python code to MySQL,
we would need to incorporate an additional MySQL connector library into our 
application\cite{mysql}.

\textit{Conclusion} [See summary of the above comparison in Table \ref{table:bret2}]:
Datastore will serve the needs of our project best as it is 
very simple to setup and manage, is simple to integrate with our application, 
and provides all the functionality we need. Despite not supporting transactions, 
Bigtable could likely be made to work, however this would just be adding more 
overhead in both setup and creating a transactions workaround. On top of this, 
Bigtable is fundamentally not the right solution for our needs. It is 
targeted at applications that store/query much more data than ours does.

\begin{table}[h]
\centering
\caption{Comparison of technologies for storing photo metadata}
\begin{tabular}{|p{1.8cm}|p{1.4cm}|p{3cm}|p{1.8cm}|p{1.8cm}|p{1.8cm}|p{1cm}|}
\cline{4-4}
\hline

% Row 1
 & Provides needed functionality? & Setup procedure & Authentication 
 & Integration with our app & Ease of setup & Secure? \\ \hline

% Row 2
Google Cloud Datastore & yes & enable Datastore in Google Cloud project
& built in: service accounts & Google Cloud Client Library for Python
& easy & yes \\ \hline

% Row 3
Google Cloud Bigtable & not natively, no support for transactions &
create Bigtable cluster, required tables & built in: service accounts 
& Google Cloud Client Library for Python & medium & yes \\ \hline

% Row 4
custom MySQL servers & yes & create virtual machines, install MySQL, 
configure network access, configure access controls, setup backup/
replication credentials file, set environment variable & requires custom setup 
& requires Python-MySQL connector & hard & yes, if done correctly \\ \hline

\end{tabular}
\label{table:bret2}
\end{table}

\subsection{Application Parallelization}

Downloading/uploading images from Google Cloud Storage and running the 
image processor on a set of images are both relatively high latency operations. 
For that reason, it is desirable for this application to do multiple operations 
concurrently. Here we consider various solutions for doing this in Python. In 
this comparison, we consider invoking the image processor binary using the 
builtin Python subprocess module. This module provides a nice Python wrapper 
around standard fork/exec/waitpid/etc. C functions.

The first solution we consider is standard multithreading using the builtin 
Python threading module. This can be used to both invoke multiple image 
processor processes concurrently and to make concurrent requests to Google 
Cloud Storage and Datastore. A common issue with use of the Python threading 
module is the Python Global Interpreter Lock (GIL)\cite{gilArticle}. The GIL 
is a global lock in the CPython interpreter that allows only one thread to 
execute code in the interpreter at a time\cite{gilArticle}. The GIL would 
not cause us many issues when invoking the image processor binary, as 
the Python subprocess implementation releases the GIL before calling 
waitpid on a particular sub-process. This means that the 
setting up/calling of a particular invocation of the image processor binary 
would be done serially, but the actual processing could be done currently as 
it is done outside of the Python interpreter and the GIL is released\cite{gilArticle}. 
The GIL is also released when making socket calls, so while setup and 
pre/post-processing of Cloud Storage uploads/downloads would be 
serialized, the socket calls themselves could be working in parallel\cite{gilArticle}. 
Implementation using the threading module requires explicit 
creation/starting/joining of Thread objects.

An alternative solution is to use the builtin Python multiprocessing.Pool 
class. This method sidesteps the GIL entirely by using processes for 
parallelism instead of threads\cite{multiproc}. This method is also desirable 
as the API is incredibly simple - a pool of n processes can be created in a 
single line and then these processes can be passed a target function and 
data in a single other line of code\cite{multiproc}. The only requirement 
here is that all parameters passed to the process pool must be picklable\cite{multiproc}. 
This would not be an issue for us as our data will take the form of standard 
Python datatypes. This approach does require creating entire new processes, 
however on Linux this can be done very very quickly. In practice, regardless 
of the application, sidestepping the GIL and implementing concurrency using 
processes instead of threads is almost always the faster solution in Python.

The last parallelization solution we investigate is applicable only to calls to 
Google Cloud Datastore. This is using batch requests instead of single requests. 
The Datastore API has built in batch request functionality which allows multiple 
queries to made in a single API call, meaning only a single HTTP request is made 
for all grouped the queries\cite{batches}. When possible, this is desirable over 
multiple concurrent requests as there is reduced overhead with establishing 
TCP connections. This application will never be sending/receiving large 
pieces of data to Datastore, so the savings associated with establishing 
far fewer TCP connections is highly desirable.

\textit{Conclusion} [See summary of the above comparison in Table \ref{table:bret3}]:
The multiprocessing.Pool class is the best solution for us 
for invoking the image processor and uploading/downloading from Google Cloud 
Storage, as it is very simple to integrate and will almost certainly offer 
higher performance than using the threading module. Using batch requests for 
Datastore calls is preferable to creating concurrent requests using the 
multiprocessing.Pool class as with batch requests we only have the overhead 
of creating a single TCP connection for all the datastore queries.

\begin{table}[h]
\centering
\caption{Comparison of technologies for application parallelism}
\begin{tabular}{|p{4.2cm}|p{4.2cm}|p{4.2cm}|}
\cline{3-3}
\hline

% Row 1
 & Applicable to & Integration Overhead \\ \hline

% Row 2
threading & image processor invocation, GCS/Datastore requests 
& medium: requires explicit thread creation/starting/joining, 
requires consciousness of shared data \\ \hline

% Row 3
multiprocessing.Pool & image processor invocation, GCS/Datastore requests
& medium: simple to invoke, requires pickleable parameters, cannot rely 
on shared data unless we use IPC \\ \hline

% Row 4
Batch requests & Datastore requests & low, requires coalescing multiple 
requests into single request  \\ \hline

\end{tabular}
\label{table:bret3}
\end{table}

% Jake's Section
\section{Eclipse Simulator}


\bibliographystyle{IEEEtran}
\bibliography{tech}

\end{document}