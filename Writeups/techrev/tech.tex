\documentclass[10pt, onecolumn, draftclsnofoot, letterpaper, compsoc]{IEEEtran}

\usepackage{cite}
\usepackage{hyperref}
%usepackage{enumitem}
\usepackage{graphicx}

\graphicspath{ {images/} }

%%%%%%%%%%%%%%%%%%%%%%%%%%%%%%%%%%%%%%%%%%%%%%%%%%%%%
% Macro for the signatures at the end                %
\newcommand*{\SignatureAndDate}[1]{%
    \par\noindent\makebox[2.5in]{\hrulefill} \hfill\makebox[2.0in]{\hrulefill}%
    \par\noindent\makebox[2.5in][l]{#1}      \hfill\makebox[2.0in][l]{Date}%
}%

\renewcommand*\contentsname{Table of Contents} % Rename ToC

\newcommand{\myindent}{\hspace{\oldparindent}}

\usepackage{cite}

% Temp title and author
\title{Requirements}
\date{\today} % Somehow this isn't working..
\author{Totality AweSun \\
		Bret~Lorimore, Jacob~Fenger, George~Harder \\
		\textit{November 4th, 2016 \\
		CS 461 - Fall 2016}}

\begin{document}

%\setlist[itemize]{topsep=1pt} % EDIT LISTS

\maketitle

% George's Section
\section{Eclipse Image Processor}

The eclipse image processor is the piece of our project that whandles the spatial
and temporal alignment of the eclipse images that are uploaded to the Eclipse
Megamovie website. We have identified three pieces\cite{imgKrista} into which
the image processor can be broken down. These are: image classification and
manipulation, the runtime environment, and circle detection. In order for this
element of our project to operate effectively it is critical that these three
pieces utilize robust and functional technologies. This section of the
technology review details what options are available for implementing each of
these three pieces, analyzes these options, and arrives at a determination as to
which option is the best in the context of this project.

% Bret's Section
\section{Image Processor Manager}

% Jake's Section
\section{Eclipse Simulator}
The eclipse simulator will be a standalone JavaScript module enabling users to
“preview” the eclipse. It will be designed in a stylized, 2D manner.
The simulator will incorporate a time slider to allow users to simulate
the eclipse in a time window spanning from 12 hours before the eclipse to 12
hours after it. As users drag the time slider, the eclipse will animate in the
simulator window. The view of the eclipse which users are presented will be
specific to a location that the user enters. Additionally, the time will be
displayed in the simulator based on what location the user enters and the
positioning of the time slider.

\subsection{User Interface}

The user interface for this web based simulator will use 2D animated depictions
of the Sun and the Moon as they appear at a user specified time and location.
Additionally, the user interface will contain background imagery such as a
city or hillside landscape. There will also be a time slider, a location
input, and a time display for user’s to interact with or view.

For the first possible method, the most basic approach would be to utilize
HTML, CSS, and JavaScript for output and input. Altering the locations of the
sun and moon would be done by repositioning images of the sun and moon based on
location data. While this approach can be simple, repositioning images and
making the interface look appealing to the user can be difficult or
inefficient.

Another method would be to utilize HTML5’s Canvas graphics API. Canvas is
capable of rendering graphics directly to the screen via JavaScript. This is
otherwise known as “immediate mode” graphics, which means that the rendered
graphics are not saved \cite{SVGvsC}. Additionally, Canvas draws individual
pixels to the screen. For this simulator, the only image objects that we need
display differently depending on user input are the Sun and Moon images.
Since these are relatively small compared to rest of the simulator,
re-rendering the Sun and Moon most likely will not result in poor performance.

The last technology for displaying the simulator would be to use another HTML5
component called Scalar Vector Graphics (SVG). This is a shape based method
for describing 2D graphics via XML.  SVG utilizes objects to describe images.
The browser is capable of re-rendering shapes automatically if attributes to
an object are changed \cite{SVG}. Adding sliders or round buttons to the
simulator can also be a straightforward approach with vector graphics.

For the user interface, speed is the most important criteria. All graphics and
other simulator resources will need to load in less than 500ms given a 1-10
Mbps internet connection. Additionally the animation used in the simulator
should be around 60 frames per second.

\begin{table}[]
\centering
\caption{My caption}
\label{my-label}
\begin{tabular}{lllll}
Method                 & Model        & Method                                                            & Difficulty of Implementation                                                  & Performance                                                                                                                       \\
Implement from Scratch & Images       & \begin{tabular}[c]{@{}l@{}}HTML,\\ JavaScript,\\ CSS\end{tabular} & \begin{tabular}[c]{@{}l@{}}May take time to start from\\ scratch\end{tabular} & \begin{tabular}[c]{@{}l@{}}Dependent upon implementation, \\ requires a lot of bandwidth to\\ send large image files\end{tabular} \\
Canvas                 & Pixel Based  & JavaScript                                                        & Built into HTML 5                                                             & \begin{tabular}[c]{@{}l@{}}Smaller surface, \\ larger number of items\end{tabular}                                                \\
SVG                    & Vector Based & \begin{tabular}[c]{@{}l@{}}JavaScript,\\ CSS\end{tabular}         & Built into HTML 5                                                             & \begin{tabular}[c]{@{}l@{}}Larger surface,\\ smaller number of items\end{tabular}
\end{tabular}
\end{table}
\bibliographystyle{IEEEtran}
\bibliography{tech}

\end{document}
